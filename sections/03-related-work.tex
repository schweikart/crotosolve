\chapter{Related work}
\label{chap:related-work}

The successful training of Quantum Machine Learning models relies on choosing an
optimization technique that can minimize the loss value effectively (see
\autoref{sec:classical-ml-intro} and \autoref{sec:optimizers}).
Most optimizers rely on the computation or approximation of the model's loss
gradient w.r.t. the model's parameters.
On the other hand, \texttt{Rotosolve} optimizes parameters of rotational Pauli
gates independently instead.
This allows it to minimize these parameters analytically instead of altering
parameters in the direction of steepest descent
\cite{ostaszewski_structure_2021}.
In this thesis, I present an extension of this method for a second type of
parameterized gates (i.e., controlled rotational Pauli gates).
Another approach has recently been presented by Wierichs et al..

In ``\emph{\citefield{wierichs_general_2022}{title}}''
\cite{wierichs_general_2022}, Wierichs et al. observe that the univariate
expected value $E(\theta)$ of a Variational Quantum Algorithm measurement can be
expressed as a finite Fourier series.

$$E(\theta) = a_0 + \sum_{l=1}^R a_l \cos(\Omega_l \theta) + b_l \sin(\Omega_l \theta)$$

Using a discrete Fourier transform, they computed the coefficients
$a_l, b_l\,, 1 \leq l \leq R$ of this expression to reconstruct $E(\theta)$ from a
finite amount of circuit evaluations.
This approach is restricted to parameterized quantum circuits that can be
expressed as a gate $U(\theta) = e^{i \theta G}$ defined by a Hermitian
generator $G$.
The $R$ frequencies $\Omega_l\,, 1 \leq l \leq R$ can be derived from the
eigenvalues of the Hermitian generator $G$.
% TODO: mention univariate stuff, see Wierichs sec. 3
% TODO: mention limitations (preprocessing?)

To my knowledge, PennyLane is currently the only major quantum computing SDK
implementing a variant of the \texttt{Rotosolve} algorithm
\cite{bergholm_pennylane_2022,ostaszewski_structure_2021}.
PennyLane's implementation extends the original \texttt{Rotosolve} approach by
using Wierichs' general method to reconstruct univariate loss functions.
This requires the calculation of the frequency spectrum $\Omega_l$ mentioned
above prior to the first iteration of the optimizer.
Only for parameters corresponding to rotational Pauli gates, PennyLane's
\texttt{Rotosolve} implementation applies the eponymous analytic methodology
from Ostaszewski's \texttt{Rotosolve} proposal.
The \texttt{Crotosolve} optimizer presented in \autoref{chap:gradient-free} can
be considered a special case of PennyLane's \texttt{Rotosolve} implementation,
which has been optimized for controlled rotational Pauli gates.

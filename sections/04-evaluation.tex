\chapter{Evaluation}
\label{chap:evaluation}

To evaluate the quality of the algorithm proposed in chapter
\ref{chap:gradient-free}, a proof-of-concept optimizer has been implemented
and benchmarked with well-known circuits.

\section{Proof of Concept}
To put the approach proposed in chapter \ref{chap:gradient-free} to test,
we are implementing a proof-of-concept application.
The application should fulfill the following requirements:

\begin{itemize}
    \item
        It should be implemented in a popular quantum computing SDK.
    \item
        It should be able to handle quantum circuits with arbitrary gates.
        Each parameter of the circuit should only be used in a single rotational
        pauli gate or in a single controlled rotational pauli gate.
        The parameters of the circuit should must be used in these gates without
        further preprocessing.
        For example $CRP(\theta)$ is a valid parameterized gate but
        $CRP(2\theta)$, $CRP(\theta - \pi)$ and $CRP(\theta^2)$ are not.
    \item
        The algorithm should perform univariate parameter optimization for the
        given circuit using \texttt{Rotosolve} \cite{ostaszewski_structure_2021}
        for rotational pauli gates and the approach developed in
        chapter \ref{chap:gradient-free} for controlled rotational pauli gates.
\end{itemize}

For the proof-of-concept implementation, we chose Pennylane
\cite{bergholm_pennylane_2022}, since it already has an implementation for the
\texttt{Rotosolve} algorithm \cite{ostaszewski_structure_2021}.

% TODO explain why not just extend Rotosolve

\section{Outline}
\begin{itemize}
    \item
        Evaluate the accuracy of the model, the number of steps in the
        optimization loop and the number of circuit evaluations
        \cite{wendenius_gradient-free_2023,ostaszewski_structure_2021}.
    \item
        Compare the results with the performance of other established
        optimizers (e.g., Adam \cite{kingma_adam_2017}, Gradient Descent and
        Quantum Natural Gradient \cite{stokes_quantum_2020}) as well as the
        \emph{Sine Exact} and \emph{Sine Iterative} variants proposed in
        \cite{wendenius_gradient-free_2023}.
        % TODO: sources for other optimizers
        For this purpose, evaluate at least the circuits from
        \cite{sim_expressibility_2019} that were also evaluated in the
        Wendenius et al. paper \cite{wendenius_gradient-free_2023}.
        % TODO: more circuits to evaluate? eileen might have access to
        %       something...
    \item
        Think about the effects of barren plateaus on this optimizer.
        % TODO: do think about this and come up with a concrete task!
        % TODO: cite barren plateaus
\end{itemize}

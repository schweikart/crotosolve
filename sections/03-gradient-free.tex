\chapter{Gradient-free \texttt{CRX}/\texttt{CRY}/\texttt{CRZ} optimization}
\label{chap:gradient-free}

\section{Outline}
\begin{itemize}
    \item
        Explain the concept of gradient-free optimization for the example of
        uncontrolled rorational gates (i.e., \texttt{RX}, \texttt{RY},
        \texttt{RZ}) from
        \cite{wendenius_gradient-free_2023,ostaszewski_structure_2021}.
    % \item TODO: we won't cite wendenius, criticize other paper
    %     Criticize \cite{wendenius_gradient-free_2023}'s incomplete
    %     analysis of the sinisoidal effect of rotational gate parameters on
    %     the measurement result.
    %     For example, the paper does not explain why the sinisoidal effect of
    %     rotational gate parameters also appears in circuits with more than
    %     one qubit.
    \item
        Explain how I plan to implement this for controlled rotational gates
        (i.e., \texttt{CRX}, \texttt{CRY}, \texttt{CRZ}).
    \item
        Express the effect of controlled rotational gate parameters on the
        expectation values mathematically and describe how the properties of
        this expression can be extracted from measurements.
\end{itemize}
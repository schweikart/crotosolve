\chapter{Gradient-free \texttt{CRX}/\texttt{CRY}/\texttt{CRZ} optimization}
\label{chap:gradient-free}

\section{Outline}
\begin{itemize}
    \item
        Explain the concept of gradient-free optimization for the example of
        uncontrolled rorational gates (i.e., \texttt{RX}, \texttt{RY},
        \texttt{RZ}) from
        \cite{wendenius_gradient-free_2023,ostaszewski_structure_2021}.
    % \item TODO: we won't cite wendenius, criticize other paper
    %     Criticize \cite{wendenius_gradient-free_2023}'s incomplete
    %     analysis of the sinisoidal effect of rotational gate parameters on
    %     the measurement result.
    %     For example, the paper does not explain why the sinisoidal effect of
    %     rotational gate parameters also appears in circuits with more than
    %     one qubit.
    \item
        Explain how I plan to implement this for controlled rotational gates
        (i.e., \texttt{CRX}, \texttt{CRY}, \texttt{CRZ}).
    \item
        Express the effect of controlled rotational gate parameters on the
        expectation values mathematically and describe how the properties of
        this expression can be extracted from measurements.
\end{itemize}

\section{Effect}
Consider a quantum circuit -- any two-bit quantum circuit -- with a
parameterized controlled pauli rotation gate.

\begin{center}
\begin{quantikz}
\lstick{\ket{0}}    & \gate[wires=2]{U} & \ctrl{1}          & \gate[wires=2]{V}\slice{$\ket{\varphi}$}  & \meter\qw \\
\lstick{\ket{0}}    &                   & \gate{RP(\theta)} & \qw                               & \qw
\end{quantikz}
\end{center}

If we measure the first qubit\footnote{
    This can in fact be any of the qubits.
    If we want to measure any other qubit, we can just append a corresponding
    swap gate to $V$ and the equation will remain the same.
}, we can describe the probability of measuring a $\ket 0$ by the following.

\begin{equation}
    \label{eq1}
    \begin{split}
        \mathbb{P}(M_0 = \ket 0)
            &= \mathbb{P}(M = \ket{00}) + \mathbb{P}(M = \ket{01}) \\
            &= \lvert\braket{00}{\varphi}\rvert^2 + \lvert\braket{01}{\varphi}\rvert^2
    \end{split}
\end{equation}

with $\ket{\varphi} = V \cdot CRP(\theta) \cdot U \cdot \ket{00}$.

To resolve this result, let's compute the more general
$\bra{\alpha} \cdot V \cdot CRP(\theta) \cdot U \cdot \ket{\beta}$.
% TODO: are the qubit indices correct?

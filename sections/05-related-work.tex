\chapter{Related work}
\label{chap:related-work}

As discussed in chapter \ref{chap:gradient-free}, our gradient-free optimization
algorithm extends the idea of the \texttt{Rotosolve}
\cite{ostaszewski_structure_2021} algorithm.

A study by Wierichs et al. \cite{wierichs_general_2022} presents a method to
reconstruct the univariate cost function for arbitrary PQCs, given the spectrum
of each individual parameter.
% TODO: only linear parameters?
% TODO: explain spectrum
The spectrum of a parameter can be used to describe the corresponding univariate
cost function with a fourier sum.
These spectra can be obtained by finding the eigenvalues of ???.
% TODO: cite articles from pennylane site

% ADOPTION
To my knowledge, Pennylane \cite{bergholm_pennylane_2022} is currently the only
major quantum computing SDK implementing a variant of the \texttt{Rotosolve}
\cite{ostaszewski_structure_2021} algorithm.
Given a PQC and the spectrum for each of its parameters, the Pennylane
\texttt{Rotosolve} implementation reconstructs the univariate cost function for
each parameter and optimizes it numerically, similar to our approach from
chapter \ref{chap:proof-of-concept}.
Only for spectra corresponding to pauli rotations, Pennylane's
\texttt{Rotosolve} implementation applies the eponymous analytic methodology
from Ostaszewski's \texttt{Rotosolve} proposal.

\section{Outline}
\begin{itemize}
    \item
        Find more literature on other gradient-free optimization
        approaches.
\end{itemize}
\chapter{Related work}
\label{chap:related-work}

As discussed in chapter \ref{chap:gradient-free}, our gradient-free optimization
algorithm extends the idea of the \texttt{Rotosolve}
\cite{ostaszewski_structure_2021} algorithm.

While in this thesis, I have presented an extension of the \texttt{Rotosolve}
algorithm for a second type of parameterized gates, a different approach with a
similar outcome has been introduced Wierichs et al..
In ``\emph{\citefield{wierichs_general_2022}{title}}'', they present\dots

TODO:
\begin{itemize}
    \item explain how they use discrete fourier series to reconstruct univariate
        loss functions
    \item Also explain what the spectrum and frequencies are in this context
    \item Name limitations of this approach (preprocessing?)
    \item Explain how pennylane adopts this into their implementation of the
        rotosolve algorithm.
    \item Note how both my approach and this one result in the same method in
        the end (at least if only RP and CRP gates are present)
\end{itemize}. . . 

% ADOPTION
To my knowledge, Pennylane \cite{bergholm_pennylane_2022} is currently the only
major quantum computing SDK implementing a variant of the \texttt{Rotosolve}
\cite{ostaszewski_structure_2021} algorithm.
Given a PQC and the spectrum for each of its parameters, the Pennylane
\texttt{Rotosolve} implementation reconstructs the univariate cost function for
each parameter and optimizes it numerically, similar to our approach from
\autoref{sec:proof-of-concept}.
Only for spectra corresponding to pauli rotations, Pennylane's
\texttt{Rotosolve} implementation applies the eponymous analytic methodology
from Ostaszewski's \texttt{Rotosolve} proposal.

\section{Outline}
\begin{itemize}
    \item
        Find more literature on other gradient-free optimization
        approaches.
    \item Generalize algorithm to detect types of gates? (see POC)
\end{itemize}
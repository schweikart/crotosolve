\chapter{Conclusion}
\label{chap:conclusion}

In this thesis, I have presented a generalization of Ostaszewski's
\texttt{Rotosolve} algorithm to controlled rotational Pauli gates.
Through mathematical analysis, I have characterized the structure of a
measurement expectation with respect to a CRP gate parameter.
I have demonstrated how targeted evaluations of the circuit with special
parameter values can be used to reconstruct this function.

Using this technique, I have implemented a proof-of-concept optimizer for
parameterized quantum circuits with parameterized rotational Pauli gates and
parameterized controlled rotational Pauli gates.
In a benchmark with circuits from \cite{sim_expressibility_2019}, I have shown
that my \texttt{Crotosolve} implementation consistently converges quicker
towards an optimal value than other optimizers.

Meanwhile, PennyLane's \texttt{Rotosolve} implementation uses results from
Wierichs et al. \cite{wierichs_general_2022} to reconstruct univariate loss
functions of arbitrary PQCs.
While \texttt{Crotosolve}'s loss curve is consistently below PennyLane's
\texttt{Rotosolve} loss curve, their implementation is more general.
It can handle PQCs where multiple gates have a common parameter and does not
rely on specialized implementations for different types of parameterized gates.

As \texttt{Rotosolve} and \texttt{Crotosolve} show competitive results compared
to gradient-based optimizers, their application should continue to be
investigated.
Future work on the approach presented in this thesis could extend the optimizer
to support more types of PQCs.
Three extensions come to mind: 1) the extension to support parameterized gates
with more than two qubits, for example the Deutsch gate $CCU(\theta)$, 2) the
extension to PQCs where multiple gates may be parameterized by the same
parameter, and 3) the extension to support gates that preprocess their
parameter, for example $CRP(5\theta^2)$.
Furthermore, a real implementation of \texttt{Crotosolve} could be improved by
removing the need to separate parameters for $RP$ and $CRP$ gates.
To remove this separation, the implementation needs to detect which parameter
belongs to which type of gate.

In PennyLane's \texttt{Rotosolve} implementation, many of these improvements
have already been incorporated through the use of general parameter-shift rules
\cite{wierichs_general_2022}.
Thus, future work could combine Wierichs' general reconstruction approach with
the optimizations presented in this thesis.
%
% further ideas:
% - combination with other types of solvers?
% - parallelization?

\chapter{Outline for the thesis}
\label{chap:outline}
% draft
%note: standard outline. add concrete description for steps i have already worked out

\begin{enumerate}
    \item Introduction
    \begin{itemize}
        \item
            Explain the motivation and relevance of the topic.
        \item
            Explicate the goals, limits and structure of the bachelor thesis.
        \item
            Name the contributions of this thesis.
        \item
            Provide a realistic outlook on the possible impacts of these
            contributions.
        \item
            A short draft for this section is already provided in
            \autoref{chap:intro}.
    \end{itemize}

    \item Background: Quantum computing and QML with PQCs
    \begin{itemize}
        \item
            Briefly explain the operation of a quantum computer.
            This section should mention qubits, gates, measurements and their
            mathematical representation.
        \item
            Explain the setup for quantum machine learning with parameterized
            quantum circuits (PQCs).
            Mention the analogy of quantum machine learning with PQCs with
            classical machine learning setups \cite{bishop_pattern_2006}.
            This section should include a few examples and cite demonstrations
            as well as evaluations of the idea.
        \item
            Go into detail on the different optimization techniques used for
            this approach.
            This section should mention state-of-the-art optimizers such as
            Adam, Gradient Descent and (Quantum) Natural Gradient.
            % TODO: cite these optimizers
            Also explain the parameter shift rule \cite{mitarai_quantum_2018} as
            we are trying to replace it.
            % TODO: think about it
    \end{itemize}

    \item Gradient-free \texttt{CRX}/\texttt{CRY}/\texttt{CRZ} optimization
    \begin{itemize}
        \item
            Explain the concept of gradient-free optimization for the example of
            uncontrolled rorational gates (i.e., \texttt{RX}, \texttt{RY},
            \texttt{RZ}) from \cite{wendenius_gradient-free_2023}.
        \item
            Explain how I plan to implement this for controlled rotational gates
            (i.e., \texttt{CRX}, \texttt{CRY}, \texttt{CRZ}).
        \item
            Do the maths.
    \end{itemize}

    \item Proof of Concent / prototype
    \begin{itemize}
        \item
            Implement a proof-of-concept using a popular quantum computing SDK.
        \item
            This POC may or may not build onto the methods proposed in
            \cite{wendenius_gradient-free_2023}, taking advantage of
            gradient-free optimization of uncontrolled rotational gates.
            % TODO: decide whether or not to do this
    \end{itemize}

    \item Evaluation
    \begin{itemize}
        \item
            Evaluate the accuracy of the model, the number of steps in the
            optimization loop and the number of circuit evaluations
            \cite{wendenius_gradient-free_2023}.
        \item
            Compare the results with the performance of other established
            optimizers (e.g., Adam, Gradient Descent and
            Quantum Natural Gradient) as well as the \emph{Sine Exact} and
            \emph{Sine Iterative} variants proposed in
            \cite{wendenius_gradient-free_2023}.
            % TODO: sources for other optimizers
        \item
            Think about the effects of barren plateaus on this optimizer.
            % TODO: do think about this and come up with a concrete task!
    \end{itemize}

    \item Related Work
    \begin{itemize}
        \item TODO
    \end{itemize}

    \item Conclusion
    \begin{itemize}
        \item
            Mention how the concept of gradient-free optimization can be
            extended from rotational gates to controlled rotational gates.
        \item
            Mention the proof-of-concept implementation and its evaluation.
        \item
            Hopefully be able to recommend 
        \item
            Present some ideas for future work.
            This might included the extension of this concept to even more
            gates.
            Unproven guess:
            If we apply this to \tt{CCX} gates, we might be able to optimize
            arbitrary circuits with this approach.
    \end{itemize}

    \item References
    \begin{itemize}
        \item
            Many of the references should already be included in
            \autoref{chap:bib} of this Exposé but more literature might
            come up in the process. 
    \end{itemize}
\end{enumerate}

\chapter{Background}
\label{chap:background}

\section{A brief introduction to Quantum Computing}
At its core, the operation of a quantum computer revolves around the gates and
qubits.

Qubits (quantum bits) make up the data register of a quantum computer.
Just like a classical bit can be in the $0$ or $1$ state, a qubit can be in the
corresponding $\ket 0$ or $\ket 1$ state.
In addition to these two basis states, however, quantum bits can be in a
superposition of these two states,

$$\ket \psi = \alpha_0 \cdot \ket 0 + \alpha_1 \cdot \ket 1,$$

where the probability amplitudes $\alpha_0, \alpha_1 \in \mathbb C$ can be any
complex numbers with $\left|\alpha_0\right|^2 + \left|\alpha_1\right|^2 = 1$.
This essentially increases the expressibility of a qubit in comparison to a
classical, digital bit from the discrete set $\left\{0, 1\right\}$ to the complex,
two-dimensional sphere surface
$\left\{\vec \alpha \in \mathbb C^2 \mid \left|\vec\alpha\right|^2\right\}$.
This means that a quantum computer with a single qubit can work with continuous,
multidimensional data, while a classical computer with a single bit can only
work with this single bit.

Unfortunalely, however, this multidimensional state of a quantum computer
collapses into a discrete state upon observation.
This means, if we try to find out what state the quantum computer is in, the
superposition will either turn into $\ket 0$ (i.e., $\alpha_0 = 1, \alpha_1 = 0$)
or $\ket 1$ (i.e., $\alpha_0 = 0, \alpha_1 = 1$).
The probability of the qubit collapsing into either of these states is given by
the squared absolute value of its probability amplitude,

$$\mathbb P(M = \ket n) = \left| \alpha_n \right|^2.$$

Thus, the normalization of the probability amplitudes, 
$\sum_{n \in \left\{0, 1\right\}} \alpha_n \ket n$, is in fact a normalization
of the probability distribution over the given basis states.
While we cannot measure $\alpha_0$ and $\alpha_1$ directly, our observation is
still influenced by them.
If we know how to reproduce a state with the same (unknown) amplitudes
$\alpha_1, \alpha_2$, we can measure the same state multiple times.
Doing so many times gives us an empirical probability distribution for the basis
states.
% GLOBAL PHASE
% BLOCH SPHERE

\subsection{Multi-qubit-systems}
yooo there are $n$ of these things rn?=??

\subsection{Gates and circuits}

Unlike with classical bits and gates, it is physically impossible for a quantum
computer to create copies of a qubit.
While classical computers typically read (copy), transform (think add) and write
data from and to registers, quantum computers have to do all operations
in-place.

\section{Outline}
Background: Quantum computing and QML with PQCs
\begin{itemize}
    \item
        Briefly explain the operation of a quantum computer
        \cite{nielsen_quantum_2007}.
        This section should mention qubits, gates, universal gate sets,
        measurements and their mathematical representation.
        Make sure to mention parameterizable gates like rotational pauli (RP)
        gates and controlled rotational pauli (CRP) gates.
    \item
        Explain the setup for quantum machine learning with parameterized
        quantum circuits (PQCs) \cite{mitarai_quantum_2018}.
        Mention the analogy of quantum machine learning with PQCs with
        classical machine learning setups \cite{bishop_pattern_2006}.
        This section should include a few examples and cite demonstrations
        as well as evaluations of the idea.
    \item
        Go into detail on the different optimization techniques used for
        this approach.
        This section should mention state-of-the-art optimizers such as
        Adam \cite{kingma_adam_2017}, Gradient Descent and
        % TODO: cite gradient descent?
        (Quantum) Natural Gradient \cite{stokes_quantum_2020}.
        % TODO: SPSA instead of QNG?
        Also explain the parameter shift rule
        \cite{mitarai_quantum_2018,schuld_evaluating_2019} as we are trying
        to replace it.
        % TODO: re-formulate
\end{itemize}

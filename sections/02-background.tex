\chapter{Background}
\label{chap:background}

\section{Outline}
Background: Quantum computing and QML with PQCs
\begin{itemize}
    \item
        Briefly explain the operation of a quantum computer
        \cite{nielsen_quantum_2007}.
        This section should mention qubits, gates, universal gate sets,
        measurements and their mathematical representation.
        Make sure to mention parameterizable gates like rotational pauli (RP)
        gates and controlled rotational pauli (CRP) gates.
    \item
        Explain the setup for quantum machine learning with parameterized
        quantum circuits (PQCs) \cite{mitarai_quantum_2018}.
        Mention the analogy of quantum machine learning with PQCs with
        classical machine learning setups \cite{bishop_pattern_2006}.
        This section should include a few examples and cite demonstrations
        as well as evaluations of the idea.
    \item
        Go into detail on the different optimization techniques used for
        this approach.
        This section should mention state-of-the-art optimizers such as
        Adam \cite{kingma_adam_2017}, Gradient Descent and
        % TODO: cite gradient descent?
        (Quantum) Natural Gradient \cite{stokes_quantum_2020}.
        % TODO: SPSA instead of QNG?
        Also explain the parameter shift rule
        \cite{mitarai_quantum_2018,schuld_evaluating_2019} as we are trying
        to replace it.
        % TODO: re-formulate
\end{itemize}

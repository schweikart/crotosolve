\chapter{\texttt{Crotosolve}: Gradient-free controlled rotational Pauli gate optimization}
\label{chap:gradient-free}

In \autoref{chap:background}, I have presented how Quantum Machine Learning
relies on optimizers to gradually minimize the expectation value of a
parameterized quantum circuit.
While many of these optimizers are gradient-based, the gradient-free
\texttt{Rotosolve} optimizer by Ostaszewski et al. treats the different
parameters of PQCs independently \cite{ostaszewski_structure_2021}.
By reconstructing the univariate loss function of a rotational Pauli gate
parameter, the minimizing parameter value can be calculated analytically.

In this chapter, I present \texttt{Crotosolve}, a similar method for parameters
of controlled rotational Pauli gates.
I first show in \autoref{sec:gradient-free:effect} that the univariate loss
curve of a controlled rotational Pauli gate can be described as the sum of two
sinusoidal functions with different frequencies.
To reproduce the concrete loss curve, the amplitudes of these functions as well
as their offsets in $x$- and $y$-direction need to be determined.
In \autoref{sec:constants}, I show how to determine these constants
algorithmically using a minimum number of circuit evaluations.
Finally, I outline how to minimize this reconstruction in
\autoref{sec:minimization}.

\section{Effect}
\label{sec:gradient-free:effect}

\begin{figure}
    \centering
    \begin{quantikz}
        \lstick{\ket{0}}    & \gate[wires=2]{U} & \ctrl{1}          & \gate[wires=2]{V}\slice[style=black]{$\ket{\varphi(\theta)}$}  & \meter\qw \\
        \lstick{\ket{0}}    &                   & \gate{RP(\theta)} & \qw                               & \qw
    \end{quantikz}
    \caption{The analyzed quantum circuit is composed of gates $U$, a controlled
    rotational Pauli gate $CRP(\theta)$, gates $V$ and a single measurement.
    $\ket{\varphi(\theta)}$ denotes the quantum state immediately before the measurement.}
    \label{fig:crp-circuit}
\end{figure}

I analyze the mathematical structure of the loss curve with respect to the
parameter $\theta$ of a single $CRP(\theta)$ gate.
Without loss of generality, I will assume a quantum circuit with only two
qubits and only a single $CRP(\theta)$ gate, as shown in
\autoref{fig:crp-circuit}.
% TODO: generality explanation
Before and after the $CRP(\theta)$ gate, the circuit can contain arbitrary gates
that do not depend on $\theta$.
These gates have been summarized as the $U$ and $V$ gates.
Following the gates, a single qubit is measured.
It should be noted that both the choice of the measured qubit and the choice of
the controlling qubit in the $CRP(\theta)$ gate are irrelevant for this
analysis, as different choices can be covered with this circuit structure by
appending or prepending swap gates to $U$ and $V$.

The possible outcomes of the measurement are distributed randomly, and this
distribution depends on $\theta$ since the circuit depends on $\theta$.
The probability of measuring $\ket 0$ in the measured qubit can be computed as
the combination of all outcomes' probabilities where the measured qubit is
$\ket 0$.

\begin{equation}
    \label{eq1}
    \begin{split}
        \mathbb{P}(M_0 = \ket 0 \mid \theta)
            &= \mathbb{P}(M = \ket{00} \mid \theta) + \mathbb{P}(M = \ket{01} \mid \theta)
    \end{split}
\end{equation}

To resolve this result, I will show in \autoref{sec:single-outcome-probability}
that there are $d_1, \dots, d_5 \in \mathbb R$ so that
$\mathbb P(M = \ket\alpha \mid \theta) = d_1 + d_3 \cos\left(\sfrac\theta2 + d_2\right) + d_5 \cos\left(\theta + d_4\right)$.
Using trigonometric identities, \autoref{eq1} can be simplified to a term
of the same structure \cite{bronstejn_taschenbuch_2016}.
% TODO: "sums of these terms"

\begin{equation}
    \label{eq:total-result-probability}
    \begin{split}
        \mathbb P\left(M_0 = \ket 0 \mid \theta\right)
            &\stackrel{(\ref{eq1})}=
                \mathbb{P}(M = \ket{00} \mid \theta) + \mathbb{P}(M = \ket{01} \mid \theta) \\
            &\stackrel{(\ref{eq:single-prob-simplified-simplified})}=
                d_1^{00} + d_3^{00} \cos\left(\sfrac\theta2 + d_2^{00}\right) + d_5^{00} \cos\left(\theta + d_4^{00}\right) \\
                &\quad + d_1^{01} + d_3^{01} \cos\left(\sfrac\theta2 + d_2^{01}\right) + d_5^{01} \cos\left(\theta + d_4^{01}\right) \\
            &= d_1 + d_3 \cos\left(\sfrac\theta2 + d_2\right) + d_5 \cos\left(\theta + d_4\right)
    \end{split}
    % TODO: janky alignment
\end{equation}

\section{Single outcome probabilities}
\label{sec:single-outcome-probability}

I will show in this section that there are constants
$d_1, \dots, d_5 \in \mathbb R$ so that the probability of a single outcome can
be expressed with the following equation.

\begin{equation}
    \mathbb P\left(M = \ket \alpha \mid \theta\right)
        = d_1 + d_3 \cos\left(\sfrac\theta2 + d_2\right) + d_5 \cos\left(\theta + d_4\right)
\end{equation}

To find this equation, I will start with the general equation for the
probability of a single measurement and the definition of the $CRP(\theta)$
gate. 
Successively, I will absorb terms that are independent from $\theta$ into
constants and combine $\sin$ and $\cos$ terms using trigonometric identities
from \cite{bronstejn_taschenbuch_2016}.
In this context, I will call terms constant if they are independent from
$\theta$.
Note that these terms are not constant w.r.t. $\ket\alpha$.

Let $\ket\alpha$ be any state from the basis of the measurement.
The probability of the measurement to result in $\ket\alpha$ can be computed
from its overlap with $\ket{\varphi(\theta)}$, as introduced in
\autoref{sec:quantum-intro}.
Here, $\ket{\varphi(\theta)} = V \cdot CRP(\theta) \cdot U\ket{00}$ denotes the
quantum state just before the measurement.

\begin{equation}
    \label{eq:single-outcome-prob}
    \begin{split}
        \mathbb P\left(M = \ket\alpha \mid \theta\right)
            &= \lvert\braket{\alpha}{\varphi(\theta)}\rvert^2\\
            &= \lvert \bra\alpha V \cdot CRP(\theta) \cdot U \ket0 \rvert^2 \\
            &= \bra\alpha V \cdot CRP(\theta) \cdot U \ket0
                \cdot \overline{\bra\alpha \cdot V \cdot CRP(\theta) \cdot U \ket0} \\
            &= \underbrace{\bra\alpha V}_{=: \bra{\tilde\alpha}} \cdot CRP(\theta)
                \cdot \underbrace{U \ket0 \cdot \bra0 \cdot U^\dagger}_{=: A} \cdot CRP(\theta)^\dagger
                \cdot \underbrace{V^\dagger \ket\alpha}_{=\ket{\tilde\alpha}} \\
            &= \bra{\tilde\alpha} CRP(\theta) \cdot A \cdot CRP(-\theta) \ket{\tilde\alpha}
    \end{split}
\end{equation}

Inserting the matrix definition of the controlled rotational Pauli gate, this
term can be multiplied out further.

\begin{equation}
    \label{eq:single-prob} 
    \begin{split}
        \mathbb P\left(M = \ket\alpha \mid \theta\right)
            &\stackrel{(\ref{eq:single-outcome-prob})}=\bra{\tilde\alpha} CRP(\theta) \cdot A \cdot CRP(-\theta) \ket{\tilde\alpha} \\
            &=\bra{\tilde\alpha}
                \left(\ket 0 \bra 0 \otimes I + \ket 1 \bra 1 \otimes RP\left(\theta\right)\right) \\
                &\quad \cdot A
                \cdot \left(\ket 0 \bra 0 \otimes I + \ket 1 \bra 1 \otimes RP\left(-\theta\right)\right)
                \ket{\tilde\alpha} \\
            &= \underbrace{\bra{\tilde\alpha} (\ket 0 \bra 0 \otimes I)}_{=: \bra\gamma} \cdot A \cdot \underbrace{(\ket 0 \bra 0 \otimes I) \ket{\tilde\alpha}}_{=: \ket{\delta}} \\
                &\quad + \underbrace{\bra{\tilde\alpha} (\ket 0 \bra 0 \otimes I)}_{= \bra\gamma} \cdot A \cdot (\ket 1 \bra 1 \otimes RP\left(-\theta\right)) \ket{\tilde\alpha} \\
                &\quad + \bra{\tilde\alpha} (\ket 1 \bra 1 \otimes RP\left(\theta\right)) \cdot A \underbrace{(\ket 0 \bra 0 \otimes I) \ket{\tilde\alpha}}_{=: \ket{\delta}} \\
                &\quad + \bra{\tilde\alpha} (\ket 1 \bra 1 \otimes RP\left(\theta\right)) \cdot A \cdot (\ket 1 \bra 1 \otimes RP\left(-\theta\right)) \ket{\tilde\alpha} \\
            &= \bra\gamma A \ket\delta \\
                &\quad + \bra\gamma A \cdot (\ket 1 \bra 1 \otimes RP\left(-\theta\right)) \ket{\tilde\alpha} \\
                &\quad + \bra{\tilde\alpha} (\ket 1 \bra 1 \otimes RP\left(\theta\right)) \cdot A \ket\delta \\
                &\quad + \bra{\tilde\alpha} (\ket 1 \bra 1 \otimes RP\left(\theta\right)) \cdot A \cdot (\ket 1 \bra 1 \otimes RP\left(-\theta\right)) \ket{\tilde\alpha} \\
    \end{split}
\end{equation}

As $\ket1\bra1 \otimes RP(\theta)$ is a block matrix, any product
$\bra x (\ket1\bra1 \otimes RP(\theta)) \ket y$ can be reduced to
$\bra{x^\downarrow} RP(\theta) \ket{y^\downarrow}$ where $\ket{x^\downarrow}$
contains the components from $\ket x$ corresponding to the non-zero matrix
block.
The summands in \autoref{eq:single-prob} can be further simplified by using this
reduction and inserting the matrix definition for $RP(\theta)$ gates, which was
presented in \autoref{eq:rotational-pauli-gates}.
With these transformations, the second summand from \autoref{eq:single-prob}
resolves to a sum of $\sin$ and $\cos$ functions of $\theta$ with frequency
$\sfrac12$.

\begin{equation}
    \label{eq:single-prob-simplification1}
    \begin{split}
            &\quad \bra\gamma A \cdot (\ket 1 \bra 1 \otimes RP\left(-\theta\right)) \ket{\tilde\alpha} \\
            &= \bra{\gamma^\downarrow} A^\downarrow \cdot RP\left(-\theta\right) \ket{\tilde\alpha^\downarrow} \\
            % TODO: this might be false
            &= \bra{\gamma^\downarrow} A^\downarrow \cdot \left(\cos\left(-\sfrac\theta2\right) I - i \sin\left(-\sfrac\theta2\right) P\right) \ket{\tilde\alpha^\downarrow} \\
            &= \cos\left(-\sfrac\theta2\right) \cdot \bra{\gamma^\downarrow} A^\downarrow \cdot I \ket{\tilde\alpha^\downarrow} \\
                &\quad - i \sin\left(-\sfrac\theta2\right) \cdot \bra{\gamma^\downarrow} A^\downarrow \cdot P \ket{\tilde\alpha^\downarrow} \\
            &= \cos\left(\sfrac\theta2\right) \cdot \underbrace{\bra{\gamma^\downarrow} A^\downarrow \cdot I \ket{\tilde\alpha^\downarrow}}_{=: c_1} \\
                &\quad + \sin\left(\sfrac\theta2\right) \cdot \underbrace{i \cdot \bra{\gamma^\downarrow} A^\downarrow \cdot P \ket{\tilde\alpha^\downarrow}}_{=: c_2} \\
            &= \cos\left(\sfrac\theta2\right) \cdot c_1 + \sin\left(\sfrac\theta2\right) \cdot c_2
    \end{split}
\end{equation}

Similarly, the third summand from \autoref{eq:single-prob} resolves to a
sum of $\sin$ and $\cos$ functions of $\theta$.
While both the second and third summands have a similar mathematical shape,
their amplitude coefficients $c_1, c_2$ and $c_3, c_4$ may be different.

\begin{equation}
    \label{eq:single-prob-simplification2}
    \begin{split}
            &\quad \bra{\tilde\alpha} (\ket 1 \bra 1 \otimes RP\left(\theta\right)) \cdot A \ket\delta \\
            &= \bra{\tilde\alpha^\downarrow} RP\left(\theta\right) \cdot A^\downarrow \ket{\delta^\downarrow} \\
            &= \bra{\tilde\alpha^\downarrow} \left(\cos\left(\sfrac\theta2\right) I - i \sin\left(\sfrac\theta2\right) P\right) \cdot A^\downarrow \ket{\delta^\downarrow} \\
            &= \cos\left(\sfrac\theta2\right) \cdot \bra{\tilde\alpha^\downarrow} I \cdot A^\downarrow \ket{\delta^\downarrow} \\
                &\quad - i \sin\left(\sfrac\theta2\right) \cdot \bra{\tilde\alpha^\downarrow} P \cdot A^\downarrow \ket{\delta^\downarrow} \\
            &= \cos\left(\sfrac\theta2\right) \cdot \underbrace{\bra{\tilde\alpha^\downarrow} I \cdot A^\downarrow \ket{\delta^\downarrow}}_{=: c_3} \\
                &\quad + \sin\left(\sfrac\theta2\right) \cdot \underbrace{\left(-i\right)\cdot \bra{\tilde\alpha^\downarrow} P \cdot A^\downarrow \ket{\delta^\downarrow}}_{=: c_4} \\
            &= \cos\left(\sfrac\theta2\right) \cdot c_3 + \sin\left(\sfrac\theta2\right) \cdot c_4
    \end{split}
\end{equation}

To resolve the fourth term, the same ideas have to be applied multiple times.
The summand is left in a form of $\sin$ and $\cos$ functions, their squares
and a product of $\sin$ and $\cos$. 

\begin{equation}
    \label{eq:single-prob-simplification3}
    \begin{split}
            &\quad \bra{\tilde\alpha} (\ket 1 \bra 1 \otimes RP\left(\theta\right)) \cdot A \cdot (\ket 1 \bra 1 \otimes RP\left(-\theta\right)) \ket{\tilde\alpha} \\
            &= \bra{\tilde\alpha^\downarrow} RP\left(\theta\right) \cdot A^\downarrow \cdot RP\left(-\theta\right) \ket{\tilde\alpha^\downarrow} \\
            &= \bra{\tilde\alpha^\downarrow} \left(\cos\left(\sfrac\theta2\right) I - i \sin\left(\sfrac\theta2\right) P\right) \\
                &\quad\cdot A^\downarrow \cdot \left(\cos\left(-\sfrac\theta2\right) I - i \sin\left(-\sfrac\theta2\right) P\right) \ket{\tilde\alpha^\downarrow} \\
            &= \cos\left(\sfrac\theta2\right)\cos\left(-\sfrac\theta2\right) \bra{\tilde\alpha^\downarrow} I \cdot A \cdot I \ket{\tilde\alpha^\downarrow} \\
                &\quad + \cos\left(\sfrac\theta2\right)\sin\left(-\sfrac\theta2\right) \cdot (-i) \cdot \bra{\tilde\alpha^\downarrow} I \cdot A \cdot P \ket{\tilde\alpha^\downarrow}  \\
                &\quad + \sin\left(\sfrac\theta2\right)\cos\left(-\sfrac\theta2\right) \cdot (-i) \cdot \bra{\tilde\alpha^\downarrow} P \cdot A \cdot I \ket{\tilde\alpha^\downarrow} \\
                &\quad + \sin\left(\sfrac\theta2\right)\sin\left(-\sfrac\theta2\right) \cdot (-i)^2 \cdot \bra{\tilde\alpha^\downarrow} P \cdot A \cdot P \ket{\tilde\alpha^\downarrow} \\
            &= \cos\left(\sfrac\theta2\right)^2 \cdot \underbrace{\bra{\tilde\alpha^\downarrow} A \ket{\tilde\alpha^\downarrow}}_{=: c_5} \\
                &\quad + \cos\left(\sfrac\theta2\right)\sin\left(\sfrac\theta2\right) \cdot \underbrace{i \cdot \bra{\tilde\alpha^\downarrow} A \cdot P \ket{\tilde\alpha^\downarrow}}_{=: c_6} \\
                &\quad + \sin\left(\sfrac\theta2\right)\cos\left(\sfrac\theta2\right) \cdot \underbrace{(-i) \cdot \bra{\tilde\alpha^\downarrow} P \cdot A \ket{\tilde\alpha^\downarrow}}_{=: c_7} \\
                &\quad + \sin\left(\sfrac\theta2\right)^2 \cdot \underbrace{\bra{\tilde\alpha^\downarrow} P \cdot A \cdot P \ket{\tilde\alpha^\downarrow}}_{=: c_8} \\
            &= \cos\left(\sfrac\theta2\right)^2 \cdot c_5 + \sin\left(\sfrac\theta2\right)^2 \cdot c_8 + \cos\left(\sfrac\theta2\right)\sin\left(\sfrac\theta2\right) \cdot \left(c_6 + c_7\right)
    \end{split}
\end{equation}

With Equations \ref{eq:single-prob-simplification1},
\ref{eq:single-prob-simplification2}, and \ref{eq:single-prob-simplification3},
\autoref{eq:single-prob} can be expressed as the following.

\begin{equation}
    \label{eq:single-prob-simplified}
    \begin{split}
        \mathbb{P}\left(M = \ket\alpha \mid \theta\right)
            &\stackrel{(\ref{eq:single-prob})}= \underbrace{\bra\gamma A \ket\delta}_{=: c_9} \\
                &\quad + \bra\gamma A \cdot (\ket 1 \bra 1 \otimes RP\left(-\theta\right)) \ket{\tilde\alpha} \\
                &\quad + \bra{\tilde\alpha} (\ket 1 \bra 1 \otimes RP\left(\theta\right)) \cdot A \ket\delta \\
                &\quad + \bra{\tilde\alpha} (\ket 1 \bra 1 \otimes RP\left(\theta\right)) \cdot A \cdot (\ket 1 \bra 1 \otimes RP\left(-\theta\right)) \ket{\tilde\alpha} \\
            &\stackrel{\substack{(\ref{eq:single-prob-simplification1})\\(\ref{eq:single-prob-simplification2})\\(\ref{eq:single-prob-simplification3})}}=
                c_9 \\
                &\quad + \cos\left(\sfrac\theta2\right) \cdot c_1 + \sin\left(\sfrac\theta2\right) \cdot c_2 \\
                &\quad + \cos\left(\sfrac\theta2\right) \cdot c_3 + \sin\left(\sfrac\theta2\right) \cdot c_4 \\
                &\quad + \cos\left(\sfrac\theta2\right)^2 \cdot c_5 + \sin\left(\sfrac\theta2\right)^2 \cdot c_8 \\
                &\quad + \cos\left(\sfrac\theta2\right)\sin\left(\sfrac\theta2\right) \cdot \left(c_6 + c_7\right) \\
            &= c_9 \\
                &\quad + \cos\left(\sfrac\theta2\right) \cdot \left(c_1 + c_3\right) + \sin\left(\sfrac\theta2\right) \cdot \left(c_2 + c_4\right) \\
                &\quad + \cos\left(\sfrac\theta2\right)^2 \cdot c_5 + \sin\left(\sfrac\theta2\right)^2 \cdot c_8 \\
                &\quad + \cos\left(\sfrac\theta2\right)\sin\left(\sfrac\theta2\right) \cdot \left(c_6 + c_7\right) \\
    \end{split}
\end{equation}

This equation is composed of many $\sin\left(\sfrac\theta2\right)$ and
$\cos\left(\sfrac\theta2\right)$ terms which occur linearly or in quadratic
form.
These products of $\sin\left(\sfrac\theta2\right)$ and
$\cos\left(\sfrac\theta2\right)$ terms can be combined into linear $\sin$ and
$\cos$ with different frequencies.
To calculate these product terms, the following trigonometric identities are
used \cite{bronstejn_taschenbuch_2016}.

\begin{subequations}
    \label{eq:trigonometric-identities}
    \begin{align}
        \cos^2\left(\theta\right)
            &= \frac12 + \frac12 \cos\left(2\theta\right)
            \label{eq:cos-squared} \\
        \sin^2\left(\theta\right)
            &= \frac12 - \frac12 \cos\left(2\theta\right)
            \label{eq:sin-squared} \\
        \cos\left(\theta\right)\sin\left(\psi\right)
            &= \frac12\sin\left(\theta + \psi\right) + \frac12 \sin\left(\theta - \psi\right)
            \label{eq:cos-sin} \\
        a\cos x + b \sin x
            &= sgn(a) \sqrt{a^2 + b^2} \cos\left(x + \arctan\left(-\frac ba\right)\right)
            \label{eq:cos-sum}
    \end{align}
\end{subequations}

Applying these identities to \autoref{eq:single-prob-simplified} leaves the
equation for the probability of a single measurement outcome in the desired
form.

\begin{equation}
    \label{eq:single-prob-simplified-simplified}
    \begin{split}
        \mathbb{P}\left(M = \ket\alpha \mid \theta\right)
            &\stackrel{(\ref{eq:single-prob-simplified})}= c_9 \\
                &\quad + \cos\left(\sfrac\theta2\right) \cdot \left(c_1 + c_3\right) + \sin\left(\sfrac\theta2\right) \cdot \left(c_2 + c_4\right) \\
                &\quad + \cos\left(\sfrac\theta2\right)^2 \cdot c_5 + \sin\left(\sfrac\theta2\right)^2 \cdot c_8 \\
                &\quad + \cos\left(\sfrac\theta2\right)\sin\left(\sfrac\theta2\right) \cdot \left(c_6 + c_7\right) \\
            &= c_9 \\
                &\quad + \cos\left(\sfrac\theta2\right) \cdot \left(c_1 + c_3\right) + \sin\left(\sfrac\theta2\right) \cdot \left(c_2 + c_4\right) \\
                &\quad + \left(\frac12 + \frac12 \cos\left(\theta\right)\right) \cdot c_5 + \left(\frac12 - \frac12 \cos\left(\theta\right)\right) \cdot c_8 \\
                &\quad + \frac12 \sin\underbrace{\left(\frac\theta2 + \frac\theta2\right)}_{=\theta} + \underbrace{\frac12 \sin\left(\frac\theta2 - \frac\theta2\right)}_{= 0} \\
            &= c_9 + \frac{c_5}{2} + \frac{c_8}{2} \\
                &\quad + \cos\left(\sfrac\theta2\right) \cdot \left(c_1 + c_3\right) + \sin\left(\sfrac\theta2\right) \cdot \left(c_2 + c_4\right) \\
                &\quad + \cos\left(\theta\right) \cdot \left(\frac{c_5}{2} - \frac{c_8}{2}\right) + \sin\left(\theta\right) \cdot \frac12 \\
            &= \underbrace{c_9 + \frac{c_5}{2} + \frac{c_8}{2}}_{=: d_1} \\
                &\quad + \cos\left(\sfrac\theta2 + \underbrace{\arctan\left(-\frac{c_2 + c_4}{c_1 + c_3}\right)}_{=: d_2}\right) \cdot \underbrace{sgn\left(c_1 + c_3\right) \sqrt{(c_1 + c_3)^2 + (c_2 + c_4)^2}}_{=: d_3} \\
                &\quad + \cos\left(\theta + \underbrace{\arctan\left(-\frac{\frac12}{\frac{c_5}{2} - \frac{c_6}{2}}\right)}_{=: d_4}\right) \cdot \underbrace{sgn\left(\frac{c_5}{2} - \frac{c_6}{2}\right) \sqrt{\left(\frac{c_5}{2} - \frac{c_6}{2}\right)^2 + \left(\frac12\right)^2}}_{=: d_5} \\
            &= d_1 + \cos\left(\sfrac\theta2 + d_2\right) \cdot d_3 + \cos\left(\theta + d_4\right) \cdot d_5
    \end{split}
\end{equation}

\section{Determining the constants}
\label{sec:constants}
\autoref{eq:total-result-probability} describes the effect of the rotation angle
parameter on the expected value of a single qubit measurement.
The simple structure of this formula comes at the cost of five unknown
constants $d_1, \dots, d_5$.
While it is possible to compute those constants from their definitions in
\autoref{eq:single-prob} - \ref{eq:single-prob-simplified-simplified},
this computation is as computationally expensive as simulating the execution of
the quantum circuit.
% TODO: add a reason for this. something about calculating <alpha|A|beta>

Instead, because of the sinusoidal nature of the function, the constants can be
determined through a few evaluations of the quantum circuit.
To simplify the syntax, I will use $y(\theta)$ to refer to the total expected
value of the measurement and $y_1(\theta), y_2(\theta)$ to refer to the two
sinusoidal functions that $y(\theta)$ is composed of.

\begin{equation}
    \label{eq:y}
    \begin{split}
        y(\theta) :&= \mathbb{P}(M_0 = \ket0 \mid \theta)\\
            &= d_1 + \underbrace{d_3 \cos(\sfrac\theta2 + d_2)}_{=: y_1(\theta)} + \underbrace{d_5 \cos(\theta + d_4)}_{=: y_2(\theta)}\\
            &= d_1 + y_1(\theta) + y_2(\theta)
    \end{split}
\end{equation}

We can analyze $y_1$ and $y_2$ separately by constructing interferences of $y$
with a phase-shifted version of itself.
Note that
$\cos(\psi + \pi) = -\cos(\psi), \cos(\psi + 2\pi) = \cos(\psi)$.

\begin{equation}
    \label{eq:d1+y2}
    \begin{split}
        y(\theta) + y(\theta + 2\pi)
            &= d_1 + d_3 \cos(\sfrac\theta2 + d_2) + d_5 \cos(\theta + d_4)\\
                &\quad + d_1 + d_3 \cos(\sfrac\theta2 + d_2 + \pi) + d_5 \cos(\theta + d_4 + 2\pi)\\
            &= d_1 + d_3 \cos(\sfrac\theta2 + d_2) + d_5 \cos(\theta + d_4)\\
                &\quad + d_1 - d_3 \cos(\sfrac\theta2 + d_2) + d_5 \cos(\theta + d_4)\\
            &= 2 d_1 + 2 d_5 \cos(\theta + d_4)\\
            &= 2 d_1 + 2 y_2(\theta)\\
        \Rightarrow y_2(\theta) &= \frac12 (y(\theta) + y(\theta + 2\pi) - 2 d_1)
    \end{split}
\end{equation}

Again, the interference of this function with itself can be used to eliminate
$y_2$.

\begin{equation}
    \label{eq:d1}
    \begin{split}
        &\quad y(\theta) + y(\theta + \pi) + y(\theta + 2\pi) + y(\theta + 3\pi)\\
            &= y(\theta) + y(\theta + 2\pi) + y(\theta + \pi) + y((\theta + \pi) + 2\pi)\\
            &\stackrel{\ref{eq:d1+y2}}= 2d_1 + 2d_5\cos(\theta + d_4) + 2d_1 + 2d_5\cos(\theta + \pi + d_4)\\
            &= 2d_1 + 2d_5\cos(\theta + d_4) + 2d_1 - 2d_5\cos(\theta + d_4)\\
            &= 4d_1\\
        \Rightarrow d_1 &= \frac14(y(\theta) + y(\theta + \pi) + y(\theta + 2\pi) + y(\theta + 3\pi))
    \end{split}
\end{equation}

With $d_1$, we can now work with $y_2$ to determine $d_4$ and $_5$.
To do so, we first need to catch an edge case.
If $d_5 = 0$, then $y_2(\theta)$ is $0$ for all angles $\theta$ and $d_4$ can be
chosen arbitrarily.
Since $\sin$ and $\cos$ have distinct zeros, we can check for this condition
with two evaluations $y_2(\theta)$ and $y_2(\theta + \sfrac32\pi)$.

\begin{equation}
    \label{eq:d5-0}
    d_5 = 0 \quad\Leftrightarrow\quad\bigwedge
    \begin{cases}
        0 = d_5 \cos(\theta + d_4) = y_2(\theta) \\
        0 = d_5 \sin(\theta + d_4) = \cos(\theta + d_4 + \sfrac32\pi) = y_2(\theta + \sfrac32\pi)
    \end{cases}
\end{equation}

If $y_2(\theta) = 0$ and $y_2(\theta + \sfrac32\pi) \neq 0$, we can derive
$d_4$ and $d_5$ as follows.
Note that we restrict the zeros of $\cos$ to be $\sfrac12\pi$ or $\sfrac32\pi$
without loss of generality since $\cos$ is $2\pi$-periodic.
We can further eliminate $\sfrac32\pi$ since choosing $\sfrac32\pi$ over
$\sfrac12\pi$ only changes the sign of the function, which can also be chosen
through its amplitude $d_5$.

\begin{equation}
    \label{eq:whateverman}
    \begin{split}
        0 &= y_2(\theta) = d_5 \cos(\theta + d_4) \\
        &\stackrel{d_5 \neq 0}\Rightarrow\quad \theta + d_4 = \sfrac12\pi\\
        &\Rightarrow\quad d_4 =\sfrac12\pi - \theta
    \end{split}
\end{equation}

The corresponding amplitude can then be computed as

\begin{equation}
    \label{eq:whateverman2}
    \begin{split}
        y_2(\theta + \sfrac32\pi)
            &= d_5 \cos(\theta + d_4 + \sfrac32\pi)\\
            &= d_5 \cos(\sfrac12\pi + \sfrac32\pi)\\
            &= d_5 \cos(2\pi)\\
            &= d_5\,.
    \end{split}
\end{equation}

If $y_2(\theta) \neq 0$, we can use the inverse of the $\tan$ function to
compute $d_4$.

\begin{equation}
    \label{eq:d4}
    \begin{split}
        \frac{y_2(\theta + \sfrac32 \pi)}{y_2(\theta)}
            &= \frac{d_5 \cos(\theta + \sfrac32 \pi + d_4)}{d_5 \cos(\theta + d_4)} \\
            &= \frac{\sin(\theta + d_4)}{\cos(\theta + d_4)} \\
            &= \tan(\theta + d_4) \\
        \Rightarrow \theta + d_4
            &= \arctan\left(\frac{y_2(\theta + \sfrac32 \pi)}{y_2(\theta)}\right) \\
        \Rightarrow d_4
            &= \arctan\left(\frac{y_2(\theta + \sfrac32 \pi)}{y_2(\theta)}\right) - \theta
            % TODO reduce to y
    \end{split}
\end{equation}

The computation of $d_5$ then needs no additional circuit evaluations.

\begin{equation}
    \label{eq:d5}
    \begin{split}
        y_2(\theta)
            &= d_5 \cos(\theta + d_4) \\
        \Rightarrow d_5
            &= \frac{y_2(\theta)}{\cos(\theta + d_4)}
            % TODO reduce to y
        % TODO: is this safe? could divide by zero but we have lots of thetas to choose from!
    \end{split}
\end{equation}

A similar approach can be used to determine the remaining constants, $d_2$ and
$d_3$.
Again, we use an interference of $y$ with a phase-shifted version of itself to
find an equation with only the missing constants.

\begin{equation}
    \label{eq:y1}
    \begin{split}
        y(\theta) - y(\theta + 2\pi)
            &= d_1 + d_3 \cos(\sfrac\theta2 + d_2) + d_5 \cos(\theta + d_4)\\
                &\quad - d_1 - d_3 \cos(\sfrac\theta2 + \pi + d_2) - d_5 \cos(\theta + 2\pi + d_4)\\
            &= d_1 + d_3 \cos(\sfrac\theta2 + d_2) + d_5 \cos(\theta + d_4)\\
                &\quad - d_1 + d_3 \cos(\sfrac\theta2 + d_2) - d_5 \cos(\theta + d_4)\\
            &= 2 d_3 \cos(\sfrac\theta2 + d_2)\\
            &= 2 y_1(\theta)\\
        \Rightarrow y_1(\theta) &= \frac12\left(y(\theta) - y(\theta + 2\pi)\right)
    \end{split}
\end{equation}

Similar to \ref{eq:d1+y2} to \ref{eq:d4}, the missing constants can be derived
from this equation.
First, we handle the case where $y_1$ zeroes out, and $d_2$ can be chosen
arbitrarily.

\begin{equation}
    \label{eq:d3-0}
    d_3 = 0 \quad\Leftrightarrow\quad \bigwedge
    \begin{cases}
        0 = d_3 \cos(\sfrac\theta2 + d_2) = y_1(\theta) \\
        0 = d_3 \sin(\sfrac\theta2 + d_2) = d_3 \cos(\sfrac\theta2 + d_2 + \sfrac32\pi) = y_1(\theta + 3\pi)
    \end{cases}
\end{equation}

Second, if $y_1(\theta) = 0$ but $y_1(\theta + 3\pi) \neq 0$, $d_4$ and $d_5$ can
be determined by choosing $\sfrac\theta2 + d_2$ as a zero point of $\cos$.

\begin{equation}
    \label{eq:whateverman-the-second}
    \begin{split}
        0 = y_1(\theta) &= d_3 \cos(\sfrac\theta2 + d_2)\\
        \stackrel{d_3 \neq 0}\Rightarrow \sfrac12\pi &= \sfrac\theta2 + d_2\\
        \Rightarrow \sfrac12\pi - \sfrac\theta2 &= d_2
    \end{split}
\end{equation}

\begin{equation}
    \label{eq:whateverman2-the-second}
    \begin{split}
        y_1(\theta + 3\pi)
            &= d_3 \cos(\sfrac\theta2 + d_2 + \sfrac32\pi)\\
            &= d_3 \cos(\sfrac12\pi + \sfrac32\pi)\\
            &= d_3 \cos(2\pi)\\
            &= d_3
    \end{split}
\end{equation}

And third, we can determine $d_2$ and $d_3$ with the use of $\tan$'s inverse if
$y_1(\theta) \neq 0$.

\begin{equation}
    \label{eq:d2}
    \begin{split}
        \frac{y_1(\theta + 3\pi)}{y_1(\theta)}
            &= \frac{d_3 \cos(\sfrac\theta2 + \sfrac32 \pi + d_2)}{d_3 \cos(\sfrac\theta2 + d_2)} \\
            &= \frac{\sin(\sfrac\theta2 + d_2)}{\cos(\sfrac\theta2 + d_2)} \\
            &= \tan(\sfrac\theta2 + d_2) \\
        \Rightarrow \sfrac\theta2 + d_2
            &= \arctan\left(\frac{y_1(\theta + 3\pi)}{y_1(\theta)}\right) \\
        \Rightarrow d_2
            &= \arctan\left(\frac{y_1(\theta + 3\pi)}{y_1(\theta)}\right) - \sfrac\theta2
    \end{split}
\end{equation}

\begin{equation}
    \label{eq:d3}
    \begin{split}
        y_1(\theta)
            &= d_3 \cos(\sfrac\theta2 + d_2) \\
        \Rightarrow d_3
            &= \frac{y_1(\theta)}{\cos(\sfrac\theta2 + d_2)}
    \end{split}
\end{equation}

Therefore, the constants $d_1, \dots, d_5$ can be determined with a total of six
quantum circuit evaluations
$y(\theta), y(\theta + \pi), y(\theta + \sfrac32\pi), y(\theta + 2\pi), y(\theta + 3\pi)$
and $y(\theta + \sfrac72\pi)$.
For convenience, they are summarized in the following.
In cases where the loss function is independent of a parameter $d_i$, the
arbitrary choice has been denoted with an asterisk $*$.

\begin{subequations}
    \label{eq:constants}
    \begin{align}
        d_1 &= \frac14 (y(\theta) + y(\theta + 2\pi) + y(\theta + \pi) + y(\theta + 3\pi))
            \label{eq:constants-d1}\\
        d_2 &=
            \begin{cases}
                *
                    \,, & \textrm{if } y_1(\theta) = 0 = y_1(\theta + 3\pi)\\
                \sfrac12\pi - \sfrac\theta2
                    \,, & \textrm{if } y_1(\theta) = 0 \neq y_1(\theta + 3\pi)\\
                \arctan\left(\frac{y_1(\theta + 3\pi)}{y_1(\theta)}\right) - \sfrac\theta2
                    \,, & \textrm{if }y_1(\theta) \neq 0
            \end{cases}
            \label{eq:constants-d2}\\
        d_3 &=
            \begin{cases}
                0
                    \,, & \textrm{if } y_1(\theta) = 0 = y_1(\theta + 3\pi)\\
                y_1(\theta + 3\pi)
                    \,, & \textrm{if } y_1(\theta) = 0 \neq y_1(\theta + 3\pi)\\
                \frac{y_1(\theta)}{\cos(\sfrac\theta2 + d_2)}
                    \,, & \textrm{if } y_1(\theta) \neq 0
            \end{cases}
            \label{eq:constants-d3}\\
        d_4 &=
            \begin{cases}
                *
                    \,, & \textrm{if } y_2(\theta) = 0 = y_2(\theta + \sfrac32\pi)\\
                \sfrac12\pi - \theta
                    \,, & \textrm{if } y_2(\theta) = 0 \neq y_2(\theta + \sfrac32\pi)\\
                \arctan\left(\frac{y_2(\theta + \sfrac32 \pi)}{y_2(\theta)}\right) - \theta
                    \,, & \textrm{if } y_2(\theta) \neq 0
            \end{cases}
            \label{eq:constants-d4}\\
        d_5 &=
            \begin{cases}
                0
                    \,, & \textrm{if } y_2(\theta) = 0 = y_2(\theta + \sfrac32\pi)\\
                y_2(\theta + \sfrac32\pi)
                    \,, & \textrm{if } y_2(\theta) = 0 \neq y_2(\theta + \sfrac32\pi)\\
                \frac{y_2(\theta)}{\cos(\theta + d_4)}
                    \,, & \textrm{if } y_2(\theta) \neq 0
            \end{cases}
            \label{eq:constants-d5}
    \end{align}
\end{subequations}

with

\begin{subequations}
    \label{eq:measurements}
    \begin{align}
        y_1(\theta) &= \frac12(y(\theta) - y(\theta + 2\pi))\\
        y_1(\theta + 3\pi) &= \frac12(y(\theta + 3\pi) - y(\theta + \pi))\\
        y_2(\theta) &= \frac12(y(\theta) + y(\theta + 2\pi) - 2d_1)\\
        y_2(\theta + \sfrac32\pi) &= \frac12(y(\theta + \sfrac32\pi) + y(\theta + \sfrac72\pi) - 2d_1)
        \,.
    \end{align}
\end{subequations}

\section{Minimizing the reconstruction}
\label{sec:minimization}

The previous sections have shown that the effect of a controlled rotational
Pauli gate parameter on the expected value of a circuit measurement can be
expressed as a sum of two sinusoidal functions.
\texttt{Crotosolve} can use this cost-efficient reconstruction to optimize a
single CRP gate parameter.
Therefore, the optimizer needs to find

\begin{equation}
    \theta_{min} = \underset{\theta \in [0, 4\pi]}{\operatorname{argmin}}\, y(\theta)\,.
\end{equation}

Since $y(\theta)$ is a univariate, real, and smooth function, a simple
minimization strategy is to use a numerical optimizer like the
\texttt{Nelder-Mead} method \cite{nelder_simplex_1965}.
As a sum of two sinusoidal functions with frequencies $1$ and $\sfrac12$,
$y(\theta)$ is $4\pi$-periodic and has either one or two local minima%
\footnote{
    If $y$ is constant, the number of minima is infinite.
    In this case, however, $\theta$ is irrelevant and any minimization result is
    optimal.}
within the $[0, 4\pi]$ bounds.
Thus, it is possible that the numerical optimizer converges to a non-global
minimum.
The chances of running into the global minimum can be increased by providing an
estimate of $\theta_{min}$ as the initial value of the minimizer.

The calculation of such an estimate could by investigated in future work.
Although it is without proof or evidence from substantial experiments, I want to
present a characteristic of the $y(\theta)$ function that I have observed to be
a helpful initial value for the minimization.
Within the boundaries of $[0, 4\pi]$, $y_1$ has a single minimum $\theta_1$ and
$y_2$ has two equal minima $\theta_{2a}, \theta_{2b}$.
Let $\theta_2$ be the value of $\theta_{2a}$ or $\theta_{2b}$ that is the
closest to $\theta_1$ and let $\theta_{min}^*$ be in the middle of $\theta_1$
and $\theta_2$.
% TODO: bounds!
Then, the global mimum of $y(\theta)$ is between $\theta_1$ and $\theta_2$
I have observed that the global minimum the global minimum of $y(\theta)$ 
The lowest minimum $x_0$ of $y$ is always located between $x_1$ and the minimum
of $y_2$ that is the closest to $x_1$.
Here, the distance and middle of two values need to take the periodicity of
$y(\theta)$ into account.
Otherwise, a global minimum near $\theta = 0$ may not be recognized.

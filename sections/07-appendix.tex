\chapter{Appendix}
\label{chap:appendix}

In section \ref{sec:optimizer-comparison}, I presented a characterization of
the \texttt{Crotosolve} loss curve in comparison to other optimizers used in
Quantum Machine Learning.
While the analysis was illustrated with a single example, this section contains
the loss curves of further PQCs.
The circuit ids referenced in the captions refer to the circuit template numbers
from \cite{sim_expressibility_2019}.
All tests use circuits with four qubits and three template layers.
The source code for the dataset generation and visualization is available with
the \texttt{Crotosolve} implementation
\cite{schweikart_schweikartcrotosolve_2023}.

\newcommand{\losscurve}[1]{%
    \subfloat[Circuit #1 with 4 qubits and 3 layers]{
        \includegraphics[width=0.5\textwidth]{loss-curve_sim#1_4x3.pdf}
    }
}

\begin{figure}
    \centering
    \losscurve{01}
    \losscurve{02}
    \hspace{0mm}
    \losscurve{03}
    \losscurve{04}
    \hspace{0mm}
    \losscurve{05}
    \losscurve{06}
    \caption{Complete evaluation results}
\end{figure}
\begin{figure}\ContinuedFloat
    \centering
    \losscurve{07}
    \losscurve{08}
    \hspace{0mm}
    \losscurve{09}
    \losscurve{10}
    \hspace{0mm}
    \losscurve{11}
    \losscurve{12}
    \caption{Complete evaluation results \emph{(cont.)}}
\end{figure}
\begin{figure}\ContinuedFloat
    \centering
    \losscurve{13}
    \losscurve{14}
    \hspace{0mm}
    \losscurve{15}
    \losscurve{16}
    \hspace{0mm}
    \losscurve{17}
    \losscurve{18}
    \caption{Complete evaluation results \emph{(cont.)}}
\end{figure}
\begin{figure}\ContinuedFloat
    \centering
    \losscurve{19}
    \caption{Complete evaluation results \emph{(cont.)}}
\end{figure}
\chapter{Proof of Concept}
\label{chap:proof-of-concept}

To put the approach proposed in chapter \ref{chap:gradient-free} to test,
we are implementing a proof-of-concept application.
The application should fulfill the following requirements:

\begin{itemize}
    \item
        It should be implemented in a popular quantum computing SDK.
    \item
        It should be able to handle quantum circuits with arbitrary gates.
        Each parameter of the circuit should only be used in a single rotational
        pauli gate or in a single controlled rotational pauli gate.
        The parameters of the circuit should must be used in these gates without
        further preprocessing.
        For example $CRP(\theta)$ is a valid parameterized gate but
        $CRP(2\theta)$, $CRP(\theta - \pi)$ and $CRP(\theta^2)$ are not.
    \item
        The algorithm should perform univariate parameter optimization for the
        given circuit using \texttt{Rotosolve} \cite{ostaszewski_structure_2021}
        for rotational pauli gates and the approach developed in
        chapter \ref{chap:gradient-free} for controlled rotational pauli gates.
\end{itemize}

\section{Outline}
\begin{itemize}
    \item
        Implement a proof-of-concept using a popular quantum computing SDK.
    \item
        This POC may or may not build onto the methods proposed in
        \cite{wendenius_gradient-free_2023,ostaszewski_structure_2021},
        taking advantage of gradient-free optimization of uncontrolled
        rotational gates.
        % TODO: decide whether or not to do this
\end{itemize}
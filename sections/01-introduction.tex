\chapter{Introduction}
\label{chap:intro}
% draft for the introduction to the thesis

%note: eileen recommends ~half a page
%todo: what is this thesis about?
%todo: why is the topic important?
%todo: what do i plan to find out and what is the solution supposed to look like?
%todo: which steps and tasks am I willing to take as part of this thesis?

While the first theoretical foundations of quantum computers have already been
developed in the 1980s, quantum computers have only recently gathered widespread
attention with the development and availability of real quantum computing
hardware \cite{nielsen_quantum_2007,hidary_quantum_2021}.
The field is currently evolving rapidly with new tools, algorithms and hardware
being released every month. % TODO: source?
Still, today's quantum devices are subject to high amounts of noise, have a very
limited number of qubits and are not fully connected, which often further
reduces the number available qubits and gates needed to perform a given task.
% TODO: source?
% TODO: given? really?

One promising idea is to apply the concept of (classical) machine learning to
quantum computers because (a) you don't need to build and understand an
algorithm by hand and (b) machine learning can benefit from a limited amount of
noise.
% TODO: "you" is informal
% TODO: citation for (b)
One common approach to quantum machine learning (QML) is to use a classical
gradient-based optimizer with a parameterized quantum circuit (PQC) to
approximate a given probability distribution.
% TODO: this enters the topic too quickly!
However, this approach requires both a lot of shots to calculate the gradient
and a lot of refinement steps to perform the gradient walk.
% TODO: is refinement really the right word for this? (same for gradient walk)

A soon-to-be-published paper by Wendenius et al.
\cite{wendenius_gradient-free_2023} explores a different optimization approach.
Instead of optimizing a circuit through a gradient-based approach, they observed
the sinusoid effect of rotational gate parameters on the output of a PQC and
calculate the optimal parameters analytically by determining the properties of
the sine wave through a small number of circuit evaluations. 
% TODO: what about the shots?
This approach has only been tested for optimizing the rotation parameter in
single-qubit rotational gates.
% TODO: that's not true, they just haven't contributed a solution for other
%       types of gates yet! There's even a plot about CRP gates!
Another important aspect to point out is that this approach does not necessarily
find the global optimum because it optimizes parameters individually.
% TODO: first part of the sentence is a waste of space
However, experiments show good results with this approach compared to
state-of-the-art optimizers like Adam, Gradient Descent and Quantum Natural
Gradient.
% TODO: cite?
These promising results naturally pose the question whether this approach can be
extended to other types of gates.

Because of their similarity with the studied rotational gates, this bachelor
thesis attempts to find a similar optimization technique for controlled
rotational gates (i.e., \texttt{CRX}, \texttt{CRY} and \texttt{CRZ}).
After some theoretical backgrounds on quantum computing and QML in general, we
will derive an analytic optimization method for parameters of controlled
rotation gates in PQCs.
We will then present a proof-of-concept implementation
\footnote{The implementation will be made available with an open-source license
on GitHub}
that extends the gradient-free optimizer by Wendenius et al..
We will evaluate the accuracy and the performance of this optimizer both
analytically and through experiments, comparing it to both the original
implementation for rotational gates and to state-of-the-art QML optimizers.
We will hopefully be able to conclude that our extension can further improve the
gradient-free QML optimizer.

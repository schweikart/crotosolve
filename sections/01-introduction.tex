\chapter{Introduction}
\label{chap:intro}
% draft for the introduction to the thesis

%note: eileen recommends ~half a page
%todo: what is this thesis about?
%todo: why is the topic important?
%todo: what do i plan to find out and what is the solution supposed to look like?
%todo: which steps and tasks am I willing to take as part of this thesis?

Quantum computing is evolving rapidly with new algorithms and hardware being
released every month. % TODO: source?
Still, today's quantum devices are subject to high amounts of noise, have a very
limited number of qubits and are not fully connected, which often further
increases the number of gates and qubits needed to perform a task.
% TODO: source?

One promising idea is to apply the concept of (classical) machine learning to
quantum computers because (a) you don't need to build and understand an
algorithm by hand and (b) machine learning can benefit from a limited amount of
noise.
% TODO: "you" is informal
% TODO: citation for (b)
One common approach to quantum machine learning (QML) is to use a classical
gradient-based optimizer with a parameterized quantum circuit (PQC) to
approximate a given probability distribution.
% TODO: this enters the topic too quickly!
However, this approach requires both a lot of shots to calculate the gradient
and a lot of refinement steps to perform the gradient walk.
% TODO: is refinement really the right word for this? (same for gradient walk)

A soon-to-be-published paper by the SCC quantum research group around
Eileen Kühn explores a different optimization approach.
% TODO: we are not the news, this is a thesis! also, we can just cite the paper.
% TODO: also, shouldn't this be wendenius et al?
Instead of optimizing a circuit through a gradient-based approach, they
calculate the optimal parameters analytically by executing the PQC three times
with different parameters in each run.
% TODO: talking about "them" seems both too distanced and personal
% TODO: what about the shots?
The approach has only been tested for optimizing the rotation parameter in
single-qubit rotational gates.
% TODO: that's not true, they just haven't contributed a solution for other
%       types of gates yet! There's even a plot about CRP gates!
Another important aspect to point out is that this approach does not necessarily
find the global optimum because it optimizes parameters individually.
% TODO: first part of the sentence is a waste of space
However, experiments show good results with this approach compared to
state-of-the-art optimizers like Adam, Gradient Descent and Quantum Natural
Gradient.
% TODO: cite?
These promising results naturally pose the question whether this approach can be
applied to other types of gates.

Because of their similarity with the studied rotational gates, this bachelor
thesis attempts to find a similar optimization technique for controlled rotational gates (i.e., \texttt{CRX},
\texttt{CRY} and \texttt{CRZ}).
After some history and theoretical backgrounds on quantum computing and quantum
machine learning in general, we will develop a derive an analytic optimization
method for parameters of controlled rotation gates in PQCs.
We will then present a proof-of-concept implementation
\footnote{The implementation will be made available with an open-source license
on GitHub}
that extends the gradient-free optimizer by Wendenius et al.
We will evaluate the accuracy and the performance of this optimizer both
analytically and through experiments, comparing it to both the Wendenius
implementation and state-of-the-art QML optimizers.
We will hopefully be able to conclude that our extension can further improve the
gradient-free QML optimizer.

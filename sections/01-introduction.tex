\chapter{Introduction}
\label{chap:intro}
% draft for the introduction to the thesis

%note: eileen recommends ~half a page
%todo: what is this thesis about?
%todo: why is the topic important?
%todo: what do i plan to find out and what is the solution supposed to look like?
%todo: which steps and tasks am I willing to take as part of this thesis?

While the first theoretical foundations of quantum computers have already been
developed in the 1980s, quantum computers have only recently gathered widespread
attention with the development and availability of real quantum computing
hardware \cite{nielsen_quantum_2007,hidary_quantum_2021}.
The field is evolving rapidly, with new tools, algorithms, and hardware being
released every month. % TODO: source?
Still, today's quantum devices are subject to high amounts of noise, have a very
limited number of qubits, and are not fully connected.
These limitations often further reduce the number of available qubits and gates
needed to perform a given task.
Researchers often refer to these limited quantum devices as
\emph{noisy intermediate-scale quantum} (NISQ) devices.
% TODO: source?
% TODO: given? really?

One promising idea for harnessing the computational power of quantum devices
within the NISQ era is to apply machine learning methodology to quantum
computers.
Machine learning can often be used to approximate complex functions without much
knowledge about their nature.
% TODO: cite! and is this too vague?
Moreover, machine learning can work with or even benefit from a limited amount
of noise \cite{ciliberto_quantum_2018}.

A typical machine learning workflow uses a parameterizable function as a
so-called model.
Machine learning aims to choose parameters for the model so that the model maps
given input data closely to given output data.
In many iterations, an \emph{optimizer} evaluates the model function to compute
the \emph{loss value}, which indicates how much the calculated value deviates
from the correct one.
% TODO: Mention data & labels?
The optimizer uses these evaluations to produce new parameter values and then
repeats its assessment.
This process is known as \emph{training}.
Many state-of-the-art optimizers use an approach called \emph{gradient descent}
to improve the model's parameters.
% TODO: explain gradient descent briefly?
Machine learning has recently made a tremendous leap due to the availability of
computational resources and improvements in model function design.
% TODO cite!
These improvements allow machine-learning applications to train billions of
parameters with terabytes of training data in many iterations.

A parameterized quantum circuit can replace the classical model function to
transfer the machine-learning methodology to quantum computers.
\emph{Parameterized quantum circuits} (PQCs) explanation here. %TODO
While gradient-based optimizers can be used for quantum machine learning,
the cost of NISQ devices does not allow for large amounts of data and large
numbers of iterations.
Additionally, each evaluation of a quantum circuit involves multiple shots to
decrease statistical noise introduced by qubit measurement.
% TODO: cite
% TODO: last sentence inconclusive

In ``\emph{\citefield{ostaszewski_structure_2021}{title}}''
\cite{ostaszewski_structure_2021}, Ostaszewski et al. explore a different
optimization approach.
They observed that the univariate expected value of a PQC w.r.t. a single
rotational Pauli gate parameter is always sinusoidal.
% TODO: this sentence is very complicated
Using three circuit evaluations with select parameter values, they could
reconstruct this univariate loss function for every parameter value.
Since the curve of sine functions is well-understood, they could then
analytically calculate the parameter value that minimizes this univariate loss
function.
The \texttt{Rotosolve} optimizer uses this approach to optimize each parameter
individually.
While univariate optimization does not guarantee convergence to the global
minimum, experiments show good results with this approach compared to
state-of-the-art optimizers like Adam \cite{kingma_adam_2017},
Gradient Descent and Quantum Natural Gradient \cite{stokes_quantum_2020}.
% TODO: cite! and remove QNG? SPSA instead?
It has only been tested for optimizing the rotation parameters and directions
for single-qubit rotational gates within arbitrary quantum circuits.
% TODO: that's not true, they just haven't contributed a solution for other
%       types of gates yet! There's even a plot about CRP gates!
% TODO: the general parameter shift rules paper does this for all PQCs already
% TODO: cite? and QNG is not a thing any more, the new paper cites SPSA

These promising results naturally pose the question of whether this approach can
be extended to other types of gates.
Because of their similarity with the studied rotational gates, this bachelor's
thesis presents \texttt{Crotosolve}, a similar optimization technique for
controlled rotational Pauli gates (i.e., \texttt{CRX}, \texttt{CRY} and
\texttt{CRZ}).

% TODO: explain results

This thesis is structured as follows.
First, in chapter \ref{chap:background}, I summarize the theoretical background
in quantum computing, machine learning, and quantum machine learning required to
understand the idea and proof behind \texttt{Crotosolve}.
Chapter \ref{chap:gradient-free} analyzes the mathematical structure of a
controlled rotational Pauli gate parameter's univariate effect on the
expected value of a quantum circuit.
It also contains a constructive proof for an algorithm that can be used to
determine the prefactors and offsets characterizing the effect function's
specific curve.
This algorithm is put to the test in chapter \ref{chap:evaluation}, where I
first present a proof-of-concept implementation\footnote{I have published all
code related to this thesis on GitHub.} of the \texttt{Crotosolve} optimizer
and then compare it to other optimizers analytically and experimentally.
% TODO further details?
Finally, in chapter \ref{chap:conclusion}, I sketch out a few ideas for future
work on this topic and conclude that ???.
% TODO insert conclusion when ready

% TODO: check again in the end
% \section{Outline}
% \begin{itemize}
%     \item
%         Explicate the goals, limits and structure of the bachelor thesis.
%     \item
%         Name the contributions of this thesis.
%     \item
%         Provide a realistic outlook on the possible impacts of these
%         contributions.
% \end{itemize}

\documentclass[
    twoside,
    english
]{sdqthesis}

% for rendering quantum circuits with tikz
\usepackage{tikz}
\usetikzlibrary{quantikz}

\usepackage{blochsphere}

% for typesetting frequently used fractions with less vertical space (\sfrac)
\usepackage{xfrac}

% for typesetting pseudocode algorithms
\usepackage[
    linesnumbered,
    ruled % algorithm styling similar to booktabs
]{algorithm2e}
%% This declares a command \Comment
%% The argument will be surrounded by /* ... */
\SetKwComment{Comment}{/* }{ */}

\usepackage[citestyle=numeric,style=numeric,backend=biber]{biblatex}
\addbibresource{thesis.bib}

\graphicspath{{./images/}}% this command from the graphics package allows to add folders (here subfolder "examplefiles") to be searched for graphic/image files

 % TODO: improve wording and appearance of the title page, maybe mention that this
 %       is an exposé for a bachelor's thesis?
 \author{Maximilian Tim Schweikart}
 \title{Extending gradient-free Optimization of Parameterized Quantum Circuits to Controlled Pauli Gates}
 \thesistype{Bachelor's Thesis}
 \myinstitute{Steinbuch Centre for Computing}
 \grouplogo{../../images/scclogo} % path is relative to the sdqthesis/logos folder!
 \reviewerone{Prof. Dr. Achim Streit}
 \reviewertwo{Prof. Dr. Bernhard Neumair}
 \advisorone{Dr. Eileen Kühn}
 \advisortwo{Dr. Max Fischer}
 \editingtime{Aug 15, 2023}{Dec 15, 2023}
 \settitle
 
\begin{document}
\setpdf
\maketitle
\frontmatter

% TODO: declaration
\setcounter{page}{1}
\pagenumbering{roman}

% TODO: abstract
%\includeabstract

\tableofcontents
% TODO: uncomment once there are any relevant figures and tables
% \listoffigures
% \listoftables

% === content of the thesis ===
\mainmatter

\chapter{Introduction}
\label{chap:intro}

While the first theoretical foundations of quantum computers have already been
developed in the 1980s, quantum computers have only recently gathered widespread
attention with the development and availability of real quantum computing
hardware \cite{nielsen_quantum_2007,hidary_quantum_2021}.
The field is evolving rapidly, with new tools, algorithms, and hardware being
released every month.
% TODO: source?
Still, today's quantum devices are subject to high amounts of noise, have a very
limited number of qubits, and are not fully connected.
These limitations often further reduce the number of available qubits and gates
needed to perform a given task \cite{cerezo_variational_2021}.
Researchers often refer to these limited quantum devices as
\emph{noisy intermediate-scale quantum} (NISQ) devices
\cite{preskill_quantum_2018}.

One promising idea for harnessing the computational power of quantum devices
within the NISQ era is to apply Machine Learning methodology to quantum
computers \cite{cerezo_variational_2021}.
% TODO: citation is for VQAs, not for QML in general!
Machine Learning can often be used to approximate complex functions without much
knowledge about their nature.
% TODO: cite! and is this too vague?
Moreover, Machine Learning can work with or even benefit from a limited amount
of noise \cite{ciliberto_quantum_2018}.

A typical Machine Learning workflow uses a parameterizable function as a
so-called model.
Machine Learning aims to choose parameters for the model so that the model maps
given input data closely to given output data.
In many iterations, an \emph{optimizer} evaluates the model function to compute
the \emph{loss value}, which indicates how much the calculated value deviates
from the correct one.
% TODO: Mention data & labels?
The optimizer uses these evaluations to produce new parameter values and then
repeats its assessment.
This process is known as \emph{training}.
Many state-of-the-art optimizers use an approach called \emph{gradient descent}
to improve the model's parameters.
% TODO: explain gradient descent briefly?
Machine Learning has recently made a tremendous leap due to the availability of
computational resources and improvements in model function design.
% TODO cite!
These improvements allow Machine-Learning applications to train billions of
parameters with terabytes of training data in many iterations.

A \emph{Parameterized Quantum Circuit} (PQC) can replace the classical model
function to transfer the machine-learning methodology to quantum computers.
PQCs are the quantum equivalent of parameterized algorithms like neural
networks.
Gradient-based optimizers can be used for Quantum Machine Learning (QML) too,
but the current price of NISQ devices does not allow for large amounts of data
and large numbers of iterations.
Additionally, each evaluation of a quantum circuit involves multiple shots to
decrease statistical noise introduced by qubit measurement.
% TODO: cite
% TODO: last sentence inconclusive

In ``\emph{\citefield{ostaszewski_structure_2021}{title}},'' Ostaszewski et al.
explore a different optimization approach \cite{ostaszewski_structure_2021}.
They observed that the univariate expected value of a PQC w.r.t. a single
rotational Pauli gate parameter is always sinusoidal.
Using three circuit evaluations with select parameter values, they could
reconstruct this univariate loss function for every parameter value.
Since the curve of sine functions is well-understood, they could then
analytically calculate the parameter value that minimizes this univariate loss
function.
The \texttt{Rotosolve} optimizer uses this approach to optimize each parameter
individually.
While univariate optimization does not guarantee convergence to the global
minimum, experiments show good results with this approach compared to
state-of-the-art optimizers like Adam \cite{kingma_adam_2017}, and SPSA
\cite{spall_multivariate_1992}.
However, the original \texttt{Rotosolve} approach was limited to optimizing
parameters of single-qubit gates.

These promising results naturally pose the question of whether this approach can
be extended to other types of gates.
Because of their similarity with the studied rotational gates, this bachelor's
thesis presents \texttt{Crotosolve}, a similar optimization technique for
controlled rotational Pauli gates.

To develop and investigate the \texttt{Crotosolve} idea, this thesis is
structured as follows.
First, in chapter \ref{chap:background}, I summarize the theoretical background
in quantum computing, Machine Learning, and Quantum Machine Learning required to
understand the idea and proof behind \texttt{Crotosolve}.
Chapter \ref{chap:gradient-free} analyzes the mathematical structure of a
controlled rotational Pauli gate parameter's univariate effect on the
expected value of a quantum circuit measurement.
It also contains a constructive proof for an algorithm that can be used to
determine the prefactors and offsets characterizing the effect function's
specific curve.

The \texttt{Crotosolve} algorithm is put to the test in chapter
\ref{chap:evaluation}.
I present a proof-of-concept implementation in PennyLane, a major quantum
computing SDK \cite{bergholm_pennylane_2022,unitary_fund_team_results_2022}.
Using this implementation and PennyLane's
\texttt{Rotosolve} \cite{ostaszewski_structure_2021},
Adam \cite{kingma_adam_2017},
Adagrad \cite{duchi_adaptive_2011}, and
Stochastic Gradient Descent implementation, I create a benchmark of their loss
curves.
% TODO: cite Stoch. GD?
The data gathered in this benchmark shows that \texttt{Crotosolve} outperforms
the other optimizers in most cases.
Its loss value progresses towards the minimum value much quicker and more
consistently in comparison to its competitors.
However, in cases with high numbers of parameters, the gradient-based Adam
optimizer shows a slightly better final result.
Even in this case, \texttt{Crotosolve} progression towards low loss values is
much steeper. 
A comprehensive discussion of this evaluation can be found in section
\ref{sec:evaluation-discussion}.

While I believe \texttt{Crotosolve} to be the only implementation that exploits
the sinusoidal structure of CRP gate loss functions explicitly, other
gradient-free approaches have been developed in the recent years.
In chapter \ref{chap:related-work}, I outline how contributions by
Wierichs et al. allow the reconstruction of univariate loss functions of almost
arbitrary PQCs \cite{wierichs_general_2022}.
These results have already been implemented in PennyLane's \texttt{Rotosolve}
implementation and were evaluated as part of chapter \ref{chap:evaluation}.

Finally, in chapter \ref{chap:conclusion}, I present a summary of the
contributions from this thesis.
Furthermore, I outline how some of the optimizations of my \texttt{Crotosolve}
implementation and the Wierichs-improved implementation of \texttt{Rotosolve}
could be fused in an optimizer that combines their advantages.
% TODO: conclusion

The proof-of-concept implementation as well as the code and benchmark data of my
evaluation are available on GitHub and Zenodo
\cite{schweikart_schweikartcrotosolve_2023}.

\chapter{Background}
\label{chap:background}

\section{A brief introduction to Quantum Computing}
At its core, the operation of a quantum computer revolves around the gates and
qubits.

Qubits (quantum bits) make up the data register of a quantum computer.
Just like a classical bit can be in the $0$ or $1$ state, a qubit can be in the
corresponding $\ket 0$ or $\ket 1$ state.
In addition to these two basis states, however, quantum bits can be in a
superposition of these two states,

$$\ket \psi = \alpha_0 \cdot \ket 0 + \alpha_1 \cdot \ket 1,$$

where the probability amplitudes $\alpha_0, \alpha_1 \in \mathbb C$ can be any
complex numbers with $\left|\alpha_0\right|^2 + \left|\alpha_1\right|^2 = 1$.
This essentially increases the expressibility of a qubit in comparison to a
classical, digital bit from the discrete set $\left\{0, 1\right\}$ to the complex,
two-dimensional sphere surface
$\left\{\vec \alpha \in \mathbb C^2 \mid \left|\vec\alpha\right|^2\right\}$.
This means that a quantum computer with a single qubit can work with continuous,
multidimensional data, while a classical computer with a single bit can only
work with this single bit.

Unfortunalely, however, this multidimensional state of a quantum computer
collapses into a discrete state upon observation.
This means, if we try to find out what state the quantum computer is in, the
superposition will either turn into $\ket 0$ (i.e., $\alpha_0 = 1, \alpha_1 = 0$)
or $\ket 1$ (i.e., $\alpha_0 = 0, \alpha_1 = 1$).
The probability of the qubit collapsing into either of these states is given by
the squared absolute value of its probability amplitude,

$$\mathbb P(M = \ket n) = \left| \alpha_n \right|^2.$$

Thus, the normalization of the probability amplitudes, 
$\sum_{n \in \left\{0, 1\right\}} \alpha_n \ket n$, is in fact a normalization
of the probability distribution over the given basis states.
While we cannot measure $\alpha_0$ and $\alpha_1$ directly, our observation is
still influenced by them.
If we know how to reproduce a state with the same (unknown) amplitudes
$\alpha_1, \alpha_2$, we can measure the same state multiple times.
Doing so many times gives us an empirical probability distribution for the basis
states.
% GLOBAL PHASE
% BLOCH SPHERE

\subsection{Multi-qubit-systems}
yooo there are $n$ of these things rn?=??

\subsection{Gates and circuits}

Unlike with classical bits and gates, it is physically impossible for a quantum
computer to create copies of a qubit.
While classical computers typically read (copy), transform (think add) and write
data from and to registers, quantum computers have to do all operations
in-place.

\section{Outline}
Background: Quantum computing and QML with PQCs
\begin{itemize}
    \item
        Briefly explain the operation of a quantum computer
        \cite{nielsen_quantum_2007}.
        This section should mention qubits, gates, universal gate sets,
        measurements and their mathematical representation.
        Make sure to mention parameterizable gates like rotational pauli (RP)
        gates and controlled rotational pauli (CRP) gates.
    \item
        Explain the setup for quantum machine learning with parameterized
        quantum circuits (PQCs) \cite{mitarai_quantum_2018}.
        Mention the analogy of quantum machine learning with PQCs with
        classical machine learning setups \cite{bishop_pattern_2006}.
        This section should include a few examples and cite demonstrations
        as well as evaluations of the idea.
    \item
        Go into detail on the different optimization techniques used for
        this approach.
        This section should mention state-of-the-art optimizers such as
        Adam \cite{kingma_adam_2017}, Gradient Descent and
        % TODO: cite gradient descent?
        (Quantum) Natural Gradient \cite{stokes_quantum_2020}.
        % TODO: SPSA instead of QNG?
        Also explain the parameter shift rule
        \cite{mitarai_quantum_2018,schuld_evaluating_2019} as we are trying
        to replace it.
        % TODO: re-formulate
\end{itemize}

\chapter{Gradient-free \texttt{CRX}/\texttt{CRY}/\texttt{CRZ} optimization}
\label{chap:gradient-free}

\section{Outline}
\begin{itemize}
    \item
        Explain the concept of gradient-free optimization for the example of
        uncontrolled rorational gates (i.e., \texttt{RX}, \texttt{RY},
        \texttt{RZ}) from
        \cite{wendenius_gradient-free_2023,ostaszewski_structure_2021}.
    % \item TODO: we won't cite wendenius, criticize other paper
    %     Criticize \cite{wendenius_gradient-free_2023}'s incomplete
    %     analysis of the sinisoidal effect of rotational gate parameters on
    %     the measurement result.
    %     For example, the paper does not explain why the sinisoidal effect of
    %     rotational gate parameters also appears in circuits with more than
    %     one qubit.
    \item
        Explain how I plan to implement this for controlled rotational gates
        (i.e., \texttt{CRX}, \texttt{CRY}, \texttt{CRZ}).
    \item
        Express the effect of controlled rotational gate parameters on the
        expectation values mathematically and describe how the properties of
        this expression can be extracted from measurements.
\end{itemize}

\section{Effect}
Consider a quantum circuit -- any two-bit quantum circuit -- with a
parameterized controlled pauli rotation gate.

\begin{center}
\begin{quantikz}
\lstick{\ket{0}}    & \gate[wires=2]{U} & \ctrl{1}          & \gate[wires=2]{V}\slice[style=black]{$\ket{\varphi}$}  & \meter\qw \\
\lstick{\ket{0}}    &                   & \gate{RP(\theta)} & \qw                               & \qw
\end{quantikz}
\end{center}

If we measure the first qubit\footnote{
    This can in fact be any of the qubits.
    If we want to measure any other qubit, we can just append a corresponding
    swap gate to $V$ and the equation will remain the same.
}, we can describe the probability of measuring a $\ket 0$ by the following.

\begin{equation}
    \label{eq1}
    \begin{split}
        \mathbb{P}(M_0 = \ket 0)
            &= \mathbb{P}(M = \ket{00}) + \mathbb{P}(M = \ket{01}) \\
            &= \lvert\braket{00}{\varphi}\rvert^2 + \lvert\braket{01}{\varphi}\rvert^2
    \end{split}
\end{equation}

with $\ket{\varphi} = V \cdot CRP(\theta) \cdot U \cdot \ket{00}$.

To resolve this result, let's compute the more general
$p_{\alpha\beta} = \bra{\alpha} \cdot V \cdot CRP(\theta) \cdot U \cdot \ket{\beta}$.
% TODO: are the qubit indices correct?

\begin{equation}
    \label{eq:single-prob}
    \begin{split}
        p_{\alpha\beta}
            &= \lvert \bra\alpha \cdot V \cdot CRP(\theta) \cdot U \ket\beta \rvert^2 \\
            &= \bra\alpha \cdot V \cdot CRP(\theta) \cdot U \ket\beta
                \cdot \overline{\bra\alpha \cdot V \cdot CRP(\theta) \cdot U \ket\beta} \\
            &= \underbrace{\bra\alpha \cdot V}_{=: \bra{\tilde\alpha}} \cdot CRP(\theta)
                \cdot \underbrace{U \ket\beta \cdot \bra\beta \cdot U^\dagger}_{=: A} \cdot CRP(\theta)^\dagger
                \cdot \underbrace{V^\dagger \ket\alpha}_{=\ket{\tilde\alpha}} \\
            &= \bra{\tilde\alpha} \cdot CRP(\theta) \cdot A \cdot CRP(-\theta) \cdot \ket{\tilde\alpha} \\
            &= \bra{\tilde\alpha}
                \cdot \left(\ket 0 \bra 0 \otimes I + \ket 1 \bra 1 \otimes RP\left(\theta\right)\right) \\
                &\quad \cdot A
                \cdot \left(\ket 0 \bra 0 \otimes I + \ket 1 \bra 1 \otimes RP\left(-\theta\right)\right)
                \cdot \ket{\tilde\alpha} \\
            &= \underbrace{\bra{\tilde\alpha} \cdot (\ket 0 \bra 0 \otimes I)}_{=: \bra\gamma} \cdot A \cdot \underbrace{(\ket 0 \bra 0 \otimes I) \ket{\tilde\alpha}}_{=: \ket{\delta}} \\
                &\quad + \underbrace{\bra{\tilde\alpha} \cdot (\ket 0 \bra 0 \otimes I)}_{= \bra\gamma} \cdot A \cdot (\ket 1 \bra 1 \otimes RP\left(-\theta\right)) \ket{\tilde\alpha} \\
                &\quad + \bra{\tilde\alpha} \cdot (\ket 1 \bra 1 \otimes RP\left(\theta\right)) \cdot A \underbrace{(\ket 0 \bra 0 \otimes I) \ket{\tilde\alpha}}_{=: \ket{\delta}} \\
                &\quad + \bra{\tilde\alpha} \cdot (\ket 1 \bra 1 \otimes RP\left(\theta\right)) \cdot A \cdot (\ket 1 \bra 1 \otimes RP\left(-\theta\right)) \ket{\tilde\alpha} \\
            &= \bra\gamma \cdot A \cdot \ket\delta \\
                &\quad + \bra\gamma \cdot A \cdot (\ket 1 \bra 1 \otimes RP\left(-\theta\right)) \ket{\tilde\alpha} \\
                &\quad + \bra{\tilde\alpha} \cdot (\ket 1 \bra 1 \otimes RP\left(\theta\right)) \cdot A \ket\delta \\
                &\quad + \bra{\tilde\alpha} \cdot (\ket 1 \bra 1 \otimes RP\left(\theta\right)) \cdot A \cdot (\ket 1 \bra 1 \otimes RP\left(-\theta\right)) \ket{\tilde\alpha} \\
    \end{split}
\end{equation}

The summands in \autoref{eq:single-prob} can be further simplified using
$RP\left(\theta\right) = \cos\left(\frac\theta2\right) I - i \sin\left(\frac\theta2\right) P$
from \cite{ostaszewski_structure_2021}, the linearity of gates and by
introducing constants for parts of the equation that are independent from
$\theta$.

\begin{equation}
    \label{eq:single-prob-simplification1}
    \begin{split}
            &\quad \bra\gamma \cdot A \cdot (\ket 1 \bra 1 \otimes RP\left(-\theta\right)) \cdot \ket{\tilde\alpha} \\
            &= \bra{\gamma^\downarrow} \cdot A^\downarrow \cdot RP\left(-\theta\right) \cdot \ket{\tilde\alpha^\downarrow} \\
            &= \bra{\gamma^\downarrow} \cdot A^\downarrow \cdot \left(\cos\left(-\sfrac\theta2\right) I - i \sin\left(-\sfrac\theta2\right) P\right) \cdot \ket{\tilde\alpha^\downarrow} \\
            &= \cos\left(-\sfrac\theta2\right) \cdot \bra{\gamma^\downarrow} \cdot A^\downarrow \cdot I \cdot \ket{\tilde\alpha^\downarrow} \\
                &\quad - i \sin\left(-\sfrac\theta2\right) \cdot \bra{\gamma^\downarrow} \cdot A^\downarrow \cdot P \cdot \ket{\tilde\alpha^\downarrow} \\
            &= \cos\left(\sfrac\theta2\right) \cdot \underbrace{\bra{\gamma^\downarrow} \cdot A^\downarrow \cdot I \cdot \ket{\tilde\alpha^\downarrow}}_{=: c_1} \\
                &\quad + \sin\left(\sfrac\theta2\right) \cdot \underbrace{i \cdot \bra{\gamma^\downarrow} \cdot A^\downarrow \cdot P \cdot \ket{\tilde\alpha^\downarrow}}_{=: c_2} \\
            &= \cos\left(\sfrac\theta2\right) \cdot c_1 + \sin\left(\sfrac\theta2\right) \cdot c_2
    \end{split}
\end{equation}

\begin{equation}
    \label{eq:single-prob-simplification2}
    \begin{split}
            &\quad \bra{\tilde\alpha} \cdot (\ket 1 \bra 1 \otimes RP\left(\theta\right)) \cdot A \cdot \ket\delta \\
            &= \bra{\tilde\alpha^\downarrow} \cdot RP\left(\theta\right) \cdot A^\downarrow \cdot \ket{\delta^\downarrow} \\
            &= \bra{\tilde\alpha^\downarrow} \cdot \left(\cos\left(\sfrac\theta2\right) I - i \sin\left(\sfrac\theta2\right) P\right) \cdot A^\downarrow \cdot \ket{\delta^\downarrow} \\
            &= \cos\left(\sfrac\theta2\right) \cdot \bra{\tilde\alpha^\downarrow} \cdot I \cdot A^\downarrow \cdot \ket{\delta^\downarrow} \\
                &\quad - i \sin\left(\sfrac\theta2\right) \cdot \bra{\tilde\alpha^\downarrow} \cdot P \cdot A^\downarrow \cdot \ket{\delta^\downarrow} \\
            &= \cos\left(\sfrac\theta2\right) \cdot \underbrace{\bra{\tilde\alpha^\downarrow} \cdot I \cdot A^\downarrow \cdot \ket{\delta^\downarrow}}_{=: c_3} \\
                &\quad + \sin\left(\sfrac\theta2\right) \cdot \underbrace{\left(-i\right)\cdot \bra{\tilde\alpha^\downarrow} \cdot P \cdot A^\downarrow \cdot \ket{\delta^\downarrow}}_{=: c_4} \\
            &= \cos\left(\sfrac\theta2\right) \cdot c_3 + \sin\left(\sfrac\theta2\right) \cdot c_4
    \end{split}
\end{equation}

\begin{equation}
    \label{eq:single-prob-simplification3}
    \begin{split}
            &\quad \bra{\tilde\alpha} \cdot (\ket 1 \bra 1 \otimes RP\left(\theta\right)) \cdot A \cdot (\ket 1 \bra 1 \otimes RP\left(-\theta\right)) \cdot \ket{\tilde\alpha} \\
            &= \bra{\tilde\alpha^\downarrow} \cdot RP\left(\theta\right) \cdot A^\downarrow \cdot RP\left(-\theta\right) \cdot \ket{\tilde\alpha^\downarrow} \\
            &= \bra{\tilde\alpha^\downarrow} \cdot \left(\cos\left(\sfrac\theta2\right) I - i \sin\left(\sfrac\theta2\right) P\right) \\
                &\quad\cdot A^\downarrow \cdot \left(\cos\left(-\sfrac\theta2\right) I - i \sin\left(-\sfrac\theta2\right) P\right) \cdot \ket{\tilde\alpha^\downarrow} \\
            &= \cos\left(\sfrac\theta2\right)\cos\left(-\sfrac\theta2\right) \bra{\tilde\alpha} \cdot I \cdot A \cdot I \cdot \ket{\tilde\alpha^\downarrow} \\
                &\quad + \cos\left(\sfrac\theta2\right)\sin\left(-\sfrac\theta2\right) \cdot (-i) \cdot \bra{\tilde\alpha} \cdot I \cdot A \cdot P \cdot \ket{\tilde\alpha^\downarrow}  \\
                &\quad + \sin\left(\sfrac\theta2\right)\cos\left(-\sfrac\theta2\right) \cdot (-i) \cdot \bra{\tilde\alpha} \cdot P \cdot A \cdot I \cdot \ket{\tilde\alpha^\downarrow} \\
                &\quad + \sin\left(\sfrac\theta2\right)\sin\left(-\sfrac\theta2\right) \cdot (-i)^2 \cdot \bra{\tilde\alpha} \cdot P \cdot A \cdot P \cdot \ket{\tilde\alpha^\downarrow} \\
            &= \cos\left(\sfrac\theta2\right)^2 \cdot \underbrace{\bra{\tilde\alpha} \cdot A \cdot \ket{\tilde\alpha^\downarrow}}_{=: c_5} \\
                &\quad + \cos\left(\sfrac\theta2\right)\sin\left(\sfrac\theta2\right) \cdot \underbrace{i \cdot \bra{\tilde\alpha} \cdot A \cdot P \cdot \ket{\tilde\alpha^\downarrow}}_{=: c_6} \\
                &\quad + \sin\left(\sfrac\theta2\right)\cos\left(\sfrac\theta2\right) \cdot \underbrace{(-i) \cdot \bra{\tilde\alpha} \cdot P \cdot A \cdot \ket{\tilde\alpha^\downarrow}}_{=: c_7} \\
                &\quad + \sin\left(\sfrac\theta2\right)^2 \cdot \underbrace{\bra{\tilde\alpha} \cdot P \cdot A \cdot P \cdot \ket{\tilde\alpha^\downarrow}}_{=: c_8} \\
            &= \cos\left(\sfrac\theta2\right)^2 \cdot c_5 + \sin\left(\sfrac\theta2\right)^2 \cdot c_8 \\
                &\quad + \cos\left(\sfrac\theta2\right)\sin\left(\sfrac\theta2\right) \cdot \left(c_6 + c_7\right)
    \end{split}
\end{equation}

With \ref{eq:single-prob-simplification1},
\ref{eq:single-prob-simplification2} and \ref{eq:single-prob-simplification3},
equation \ref{eq:single-prob} can be expressed as the following.

\begin{equation}
    \label{eq:single-prob-simplified}
    \begin{split}
        p_{\alpha\beta}
            &\stackrel{(\ref{eq:single-prob})}= \underbrace{\bra\gamma \cdot A \cdot \ket\delta}_{=: c_9} \\
                &\quad + \bra\gamma \cdot A \cdot (\ket 1 \bra 1 \otimes RP\left(-\theta\right)) \ket{\tilde\alpha} \\
                &\quad + \bra{\tilde\alpha} \cdot (\ket 1 \bra 1 \otimes RP\left(\theta\right)) \cdot A \ket\delta \\
                &\quad + \bra{\tilde\alpha} \cdot (\ket 1 \bra 1 \otimes RP\left(\theta\right)) \cdot A \cdot (\ket 1 \bra 1 \otimes RP\left(-\theta\right)) \ket{\tilde\alpha} \\
            &\stackrel{\substack{(\ref{eq:single-prob-simplification1})\\(\ref{eq:single-prob-simplification2})\\(\ref{eq:single-prob-simplification3})}}=
                c_9 \\
                &\quad + \cos\left(\sfrac\theta2\right) \cdot c_1 + \sin\left(\sfrac\theta2\right) \cdot c_2 \\
                &\quad + \cos\left(\sfrac\theta2\right) \cdot c_3 + \sin\left(\sfrac\theta2\right) \cdot c_4 \\
                &\quad + \cos\left(\sfrac\theta2\right)^2 \cdot c_5 + \sin\left(\sfrac\theta2\right)^2 \cdot c_8 \\
                &\quad + \cos\left(\sfrac\theta2\right)\sin\left(\sfrac\theta2\right) \cdot \left(c_6 + c_7\right) \\
            &= c_9 \\
                &\quad + \cos\left(\sfrac\theta2\right) \cdot \left(c_1 + c_3\right) + \sin\left(\sfrac\theta2\right) \cdot \left(c_2 + c_4\right) \\
                &\quad + \cos\left(\sfrac\theta2\right)^2 \cdot c_5 + \sin\left(\sfrac\theta2\right)^2 \cdot c_8 \\
                &\quad + \cos\left(\sfrac\theta2\right)\sin\left(\sfrac\theta2\right) \cdot \left(c_6 + c_7\right) \\
    \end{split}
\end{equation}

This equation in turn can be simplified even further through the use of the
following trigonometric identities:

\begin{equation}
    \begin{split}
        \cos^2\left(\theta\right) &= \frac12 + \frac12 \cos\left(2\theta\right) \\
        \sin^2\left(\theta\right) &= \frac12 - \frac12 \cos\left(2\theta\right) \\
        \cos\left(\theta\right)\sin\left(\varphi\right) &= \frac12\sin\left(\theta + \varphi\right) + \frac12 \sin\left(\theta - \varphi\right) \\
        a\cos x + b \sin x &= sgn(a) \sqrt{a^2 + b^2} \cos\left(x + \arctan\left(-\frac ba\right)\right)
    \end{split}
\end{equation}

\begin{equation}
    \label{eq:single-prob-simplified-simplified}
    \begin{split}
        p_{\alpha\beta}
            &\stackrel{(\ref{eq:single-prob-simplified})}= c_9 \\
                &\quad + \cos\left(\sfrac\theta2\right) \cdot \left(c_1 + c_3\right) + \sin\left(\sfrac\theta2\right) \cdot \left(c_2 + c_4\right) \\
                &\quad + \cos\left(\sfrac\theta2\right)^2 \cdot c_5 + \sin\left(\sfrac\theta2\right)^2 \cdot c_8 \\
                &\quad + \cos\left(\sfrac\theta2\right)\sin\left(\sfrac\theta2\right) \cdot \left(c_6 + c_7\right) \\
            &= c_9 \\
                &\quad + \cos\left(\sfrac\theta2\right) \cdot \left(c_1 + c_3\right) + \sin\left(\sfrac\theta2\right) \cdot \left(c_2 + c_4\right) \\
                &\quad + \left(\frac12 + \frac12 \cos\left(\theta\right)\right) \cdot c_5 + \left(\frac12 - \frac12 \cos\left(\theta\right)\right) \cdot c_8 \\
                &\quad + \frac12 \sin\left(\underline{\frac\theta2 + \frac\theta2}_{=\theta}\right) + \underbrace{\frac12 \sin\left(\frac\theta2 - \frac\theta2\right)}_{= 0} \\
            &= c_9 + \frac{c_5}{2} + \frac{c_8}{2} \\
                &\quad + \cos\left(\sfrac\theta2\right) \cdot \left(c_1 + c_3\right) + \sin\left(\sfrac\theta2\right) \cdot \left(c_2 + c_4\right) \\
                &\quad + \cos\left(\theta\right) \cdot \left(\frac{c_5}{2} - \frac{c_8}{2}\right) + \sin\left(\theta\right) \cdot \frac12 \\
            &= \underbrace{c_9 + \frac{c_5}{2} + \frac{c_8}{2}}_{=: d_1} \\
                &\quad + \cos\left(\sfrac\theta2 + \underbrace{\arctan\left(-\frac{c_2 + c_4}{c_1 + c_3}\right)}_{=: d_2}\right) \cdot \underbrace{sgn\left(c_1 + c_3\right) \sqrt{(c_1 + c_3)^2 + (c_2 + c_4)^2}}_{=: d_3} \\
                &\quad + \cos\left(\theta + \underbrace{\arctan\left(-\frac{\frac12}{\frac{c_5}{2} - \frac{c_6}{2}}\right)}_{=: d_4}\right) \cdot \underbrace{sgn\left(\frac{c_5}{2} - \frac{c_6}{2}\right) \sqrt{\left(\frac{c_5}{2} - \frac{c_6}{2}\right)^2 + \left(\frac12\right)^2}}_{=: d_5} \\
            &= d_1 + \cos\left(\sfrac\theta2 + d_2\right) \cdot d_3 + \cos\left(\theta + d_4\right) \cdot d_5
    \end{split}
\end{equation}

\section{Determining the constants}
Equation \ref{eq:single-prob-simplified-simplified} describes the effect of the
rotation angle parameter on the expectation value of a single qubit.
The simple structure of this formula comes at the cost of five unknown
constants $d_1, \dots, d_5$.
While it is possible to compute those constants from their definitions in
equations \ref{eq:single-prob} - \ref{eq:single-prob-simplified-simplified},
this computation is as computationally expensive as simulating the execution of
the quantum circuit.
% TODO: add a reason for this. something about calculating <alpha|A|beta>

Instead, because of the sinoidal nature of the function, the constants can be
determined through a few evaluations of the quantum circuit.

\begin{equation}
    \label{eq:y}
    \begin{split}
        y(\theta)
            &= d_1 + \underbrace{d_3 \cos(\sfrac\theta2 + d_2)}_{=: y_1(\theta)} + \underbrace{d_5 \cos(\theta + d_4)}_{=: y_2(\theta)}\\
            &= d_1 + y_1(\theta) + y_2(\theta)
    \end{split}
\end{equation}

We can analyze $y_1$ and $y_2$ separately by constructing interferences of $y$
with a phase-shifted version of itself.
Note that
$\cos(\varphi + \pi) = -\cos(\varphi), \cos(\varphi + 2\pi) = \cos(\varphi)$.

\begin{equation}
    \label{eq:d1+y2}
    \begin{split}
        y(\theta) + y(\theta + 2\pi)
            &= d_1 + d_3 \cos(\sfrac\theta2 + d_2) + d_5 \cos(\theta + d_4)\\
                &\quad + d_1 + d_3 \cos(\sfrac\theta2 + d_2 + \pi) + d_5 \cos(\theta + d_4 + 2\pi)\\
            &= d_1 + d_3 \cos(\sfrac\theta2 + d_2) + d_5 \cos(\theta + d_4)\\
                &\quad + d_1 - d_3 \cos(\sfrac\theta2 + d_2) + d_5 \cos(\theta + d_4)\\
            &= 2 d_1 + 2 d_5 \cos(\theta + d_4)\\
            &= 2 d_1 + 2 y_2(\theta)\\
        \Rightarrow y_2(\theta) &= \frac12 (y(\theta) + y(\theta + 2\pi) - 2 d_1)
    \end{split}
\end{equation}

Again, the interference of this function with itself can be used to eliminate
$y_2$.

\begin{equation}
    \label{eq:d1}
    \begin{split}
        &\quad y(\theta) + y(\theta + \pi) + y(\theta + 2\pi) + y(\theta + 3\pi)\\
            &= y(\theta) + y(\theta + 2\pi) + y(\theta + \pi) + y((\theta + \pi) + 2\pi)\\
            &\stackrel{\ref{eq:d1+y2}}= 2d_1 + 2d_5\cos(\theta + d_4) + 2d_1 + 2d_5\cos(\theta + \pi + d_4)\\
            &= 2d_1 + 2d_5\cos(\theta + d_4) + 2d_1 - 2d_5\cos(\theta + d_4)\\
            &= 4d_1\\
        \Rightarrow d_1 &= \frac14(y(\theta) + y(\theta + \pi) + y(\theta + 2\pi) + y(\theta + 3\pi))
    \end{split}
\end{equation}

With $d_1$, we can now work with $y_2$ to determine $d_4$ and $_5$.
To do so, we first need to catch an edge case.
If $d_5 = 0$, then $y_2(\theta)$ is $0$ for all angles $\theta$ and $d_4$ can be
chosen arbitrarily.
Since $\sin$ and $\cos$ have distinct zeros, we can check for this condition
with two evaluations $y_2(\theta)$ and $y_2(\theta + \sfrac32\pi)$.

\begin{equation}
    \label{eq:d5-0}
    d_5 = 0 \quad\Leftrightarrow\quad\bigwedge
    \begin{cases}
        0 = d_5 \cos(\theta + d_4) = y_2(\theta) \\
        0 = d_5 \sin(\theta + d_4) = \cos(\theta + d_4 + \sfrac32\pi) = y_2(\theta + \sfrac32\pi)
    \end{cases}
\end{equation}

If $y_2(\theta) = 0$ and $y_2(\theta + \sfrac32\pi) \neq 0$, we can derive
$d_4$ and $d_5$ as follows.
Note that we restrict the zeros of $\cos$ to be $\sfrac12\pi$ or $\sfrac32\pi$
without loss of generality since $\cos$ is $2\pi$-periodic.
We can further eliminate $\sfrac32\pi$ since choosing $\sfrac32\pi$ over
$\sfrac12\pi$ only changes the sign of the function, which can also be chosen
through its amplitude $d_5$.

\begin{equation}
    \label{eq:whateverman}
    \begin{split}
        0 &= y_2(\theta) = d_5 \cos(\theta + d_4) \\
        &\stackrel{d_5 \neq 0}\Rightarrow\quad \theta + d_4 = \sfrac12\pi\\
        &\Rightarrow\quad d_4 =\sfrac12\pi - \theta
    \end{split}
\end{equation}

The corresponding amplitude can then be computed as

\begin{equation}
    \label{eq:whateverman2}
    \begin{split}
        y_2(\theta + \sfrac32\pi)
            &= d_5 \cos(\theta + d_4 + \sfrac32\pi)\\
            &= d_5 \cos(\sfrac12\pi + \sfrac32\pi)\\
            &= d_5 \cos(2\pi)\\
            &= d_5\,.
    \end{split}
\end{equation}

If $y_2(\theta) \neq 0$, we can use the inverse of the $\tan$ function to
compute $d_4$.

\begin{equation}
    \label{eq:d4}
    \begin{split}
        \frac{y_2(\theta + \sfrac32 \pi)}{y_2(\theta)}
            &= \frac{d_5 \cos(\theta + \sfrac32 \pi + d_4)}{d_5 \cos(\theta + d_4)} \\
            &= \frac{\sin(\theta + d_4)}{\cos(\theta + d_4)} \\
            &= \tan(\theta + d_4) \\
        \Rightarrow \theta + d_4
            &= \arctan\left(\frac{y_2(\theta + \sfrac32 \pi)}{y_2(\theta)}\right) \\
        \Rightarrow d_4
            &= \arctan\left(\frac{y_2(\theta + \sfrac32 \pi)}{y_2(\theta)}\right) - \theta
            % TODO reduce to y
    \end{split}
\end{equation}

The computation of $d_5$ then needs no additional circuit evaluations.

\begin{equation}
    \label{eq:d5}
    \begin{split}
        y_2(\theta)
            &= d_5 \cos(\theta + d_4) \\
        \Rightarrow d_5
            &= \frac{y_2(\theta)}{\cos(\theta + d_4)}
            % TODO reduce to y
        % TODO: is this safe? could divide by zero but we have lots of thetas to choose from!
    \end{split}
\end{equation}

A similar approach can be used to determine the remaining constants, $d_2$ and
$d_3$.
Again, we use an interference of $y$ with a phase-shifted version of itself to
find an equation with only the missing constants.

\begin{equation}
    \label{eq:y1}
    \begin{split}
        y(\theta) - y(\theta + 2\pi)
            &= d_1 + d_3 \cos(\sfrac\theta2 + d_2) + d_5 \cos(\theta + d_4)\\
                &\quad - d_1 - d_3 \cos(\sfrac\theta2 + \pi + d_2) - d_5 \cos(\theta + 2\pi + d_4)\\
            &= d_1 + d_3 \cos(\sfrac\theta2 + d_2) + d_5 \cos(\theta + d_4)\\
                &\quad - d_1 + d_3 \cos(\sfrac\theta2 + d_2) - d_5 \cos(\theta + d_4)\\
            &= 2 d_3 \cos(\sfrac\theta2 + d_2)\\
            &= 2 y_1(\theta)\\
        \Rightarrow y_1(\theta) &= \frac12\left(y(\theta) - y(\theta + 2\pi)\right)
    \end{split}
\end{equation}

And, again similary to \ref{eq:d1+y2} to \ref{eq:d4}, the missing constants can
be derived from this equation.
First, we handle the case where $y_1$ zeroes out and $d_2$ can be chosen
arbitrarily.

\begin{equation}
    \label{eq:d3-0}
    d_3 = 0 \quad\Leftrightarrow\quad \bigwedge
    \begin{cases}
        0 = d_3 \cos(\sfrac\theta2 + d_2) = y_1(\theta) \\
        0 = d_3 \sin(\sfrac\theta2 + d_2) = d_3 \cos(\sfrac\theta2 + d_2 + \sfrac32\pi) = y_1(\theta + 3\pi)
    \end{cases}
\end{equation}

Second, if $y_1(\theta) = 0$ but $y_1(\theta + 3\pi) \neq 0$, $d_4$ and $d_5$ can
be determined by choosing $\sfrac\theta2 + d_2$ as a zero point of $\cos$.

\begin{equation}
    \label{eq:whateverman-the-second}
    \begin{split}
        0 = y_1(\theta) &= d_3 \cos(\sfrac\theta2 + d_2)\\
        \stackrel{d_3 \neq 0}\Rightarrow \sfrac12\pi &= \sfrac\theta2 + d_2\\
        \Rightarrow \sfrac12\pi - \sfrac\theta2 &= d_2
    \end{split}
\end{equation}

\begin{equation}
    \label{eq:whateverman2-the-second}
    \begin{split}
        y_1(\theta + 3\pi)
            &= d_3 \cos(\sfrac\theta2 + d_2 + \sfrac32\pi)\\
            &= d_3 \cos(\sfrac12\pi + \sfrac32\pi)\\
            &= d_3 \cos(2\pi)\\
            &= d_3
    \end{split}
\end{equation}

And third, we can determine $d_2$ and $d_3$ with the use of $\tan$'s inverse if
$y_1(\theta) \neq 0$.

\begin{equation}
    \label{eq:d2}
    \begin{split}
        \frac{y_1(\theta + 3\pi)}{y_1(\theta)}
            &= \frac{d_3 \cos(\sfrac\theta2 + \sfrac32 \pi + d_2)}{d_3 \cos(\sfrac\theta2 + d_2)} \\
            &= \frac{\sin(\sfrac\theta2 + d_2)}{\cos(\sfrac\theta2 + d_2)} \\
            &= \tan(\sfrac\theta2 + d_2) \\
        \Rightarrow \sfrac\theta2 + d_2
            &= \arctan\left(\frac{y_1(\theta + 3\pi)}{y_1(\theta)}\right) \\
        \Rightarrow d_2
            &= \arctan\left(\frac{y_1(\theta + 3\pi)}{y_1(\theta)}\right) - \sfrac\theta2
            % TODO resolve for y
    \end{split}
\end{equation}

\begin{equation}
    \label{eq:d3}
    \begin{split}
        y_1(\theta)
            &= d_3 \cos(\sfrac\theta2 + d_2) \\
        \Rightarrow d_3
            &= \frac{y_1(\theta)}{\cos(\sfrac\theta2 + d_2)}
            % TODO: resolve for y
    \end{split}
\end{equation}

Therefore, the constants $d_1, \dots, d_5$ can be determined with a total of six
quantum circuit evaluations
$y(\theta), y(\theta + \pi), y(\theta + \sfrac32\pi), y(\theta + 2\pi), y(\theta + 3\pi)$
and $y(\theta + \sfrac72\pi)$.
For convenience, they are summarized in the following.

\begin{subequations}
    \label{eq:constants}
    \begin{align}
        d_1 &= \frac14 (y(\theta) + y(\theta + 2\pi) + y(\theta + \pi) + y(\theta + 3\pi))
            \label{eq:constants-d1}\\
        d_4 &= \arctan\left(\frac{y(\theta + \sfrac32 \pi) + y(\theta + \sfrac72 \pi) - 2 d_1}{y(\theta) + y(\theta + 2\pi) - 2 d_1}\right) - \theta
            \label{eq:constants-d4}\\
        d_5 &= \frac{y(\theta) + y(\theta + 2\pi) - 2 d_1}{2 \cos(\theta + d_4)}
            \label{eq:constants-d5}\\
        d_2 &= \arctan\left(\frac{y(\theta + 3\pi) - y(\theta + \pi)}{y(\theta) - y(\theta - 2\pi)}\right) - \sfrac\theta2
            \label{eq:constants-d2}\\
        d_3 &= \frac{y(\theta) - y(\theta - 2\pi)}{2 \cos(\sfrac\theta2 + d_2)}
            \label{eq:constants-d3}
            %TODO reorder
    \end{align}
\end{subequations}

\chapter{Evaluation}
\label{chap:evaluation}

To evaluate the quality of the algorithm proposed in chapter
\ref{chap:gradient-free}, we implement a proof-of-concept optimizer.
We compare our optimizer to a number of other optimizers frequently used in QML
by creating a loss curve benchmark for all optimizers.
The benchmark features parameterizable quantum circuits with various degrees of
expressibility and from various research applications.

\section{Proof of Concept}
To put the approach proposed in chapter \ref{chap:gradient-free} to test,
we implement a proof-of-concept application with PennyLane
\cite{bergholm_pennylane_2022}.
PennyLane is a popular \cite{unitary_fund_team_results_2022} quantum computing
SDK with extensive documentation and a number of QML optimizers already
implemented.
Particularly, PennyLane is currently the only quantum SDK to implement the
\texttt{Rotosolve} optimizer, to which we want to compare our optimizer in
section \ref{sec:optimizer-comparison}.

While the circuits we want to optimize can be composed of arbitrary static
(i.e., non-parameterized) gates, we limit the types of parameterizable gates to
rotational pauli gates and controlled rotational pauli gates.
Additionally, we assume all parameters are independent from each other and used
without any preprocessing (e.g., having a $RP(x^2)$ gate for a parameter $x$).
% TODO comment on whether this is an actual limitation?

For our algorithm to work, we also need to know which parameters belong to which
type of parameterized gate.
As a simple solution, we assume all $RP$ parameters are passed as an array
separate from the $CRP$ parameters array.
It is in theory, however, also possible to detect the type of gate through static analysis
of the circuit, which is beyond the scope of this thesis.

Similar to the \texttt{Rotosolve} algorithm \cite{ostaszewski_structure_2021},
our \texttt{Crotosolve} implementation optimizes each parameter individually.
Each parameter corresponding to an uncontrolled rotational pauli gate can be
optimized following the exact approach presented by Ostaszewski et al..
And for each parameter corresponding to a controlled rotational pauli gate, we
first reconstruct the univariate cost function (see section \ref{sec:constants})
and then use a numerical optimizer to find the minimizing parameter value (see
section \ref{sec:minimization}).
The algorithm pseudocode is stated in listing \ref{alg:crotosolve}.

\begin{algorithm}
    \caption{The \texttt{Crotosolve} algorithm updates parameter individually}
    \label{alg:crotosolve}
    %
    \SetKwFunction{ReconstructCrp}{ReconstructCrp}
    \SetKwFunction{RotosolveUpdate}{RotosolveUpdate}
    \SetKwFunction{MakeUnivariate}{MakeUnivariate}
    %
    \KwData{$prev\_params \in \mathbb R^n$, $circuit$}
    \KwResult{$params \in \mathbb R^n$}
    \BlankLine
    $params \gets copy(prev\_params)$\;
    \For{$p \gets 1$ \KwTo $n$}{
        $uni \gets $ \MakeUnivariate{$circuit$, $params$, $p$}\;
        \eIf{$p$ belongs to controlled gate}{
            $recon \gets $ \ReconstructCrp{$uni$} \Comment*[r]{see section \ref{sec:constants}}
            $next\_params[p] \gets \underset{x \in [0, 4\pi]}{\operatorname{argmin}}\, recon(x)$ \Comment*[r]{using numerical minimizer}
        }{
            $next\_params[p] \gets $ \RotosolveUpdate{$uni$}\;
        }
    }
\end{algorithm}

\section{Optimizer comparison}
\label{sec:optimizer-comparison}
\subsubsection*{Methodology}

We want to examine \texttt{Crotosolve}'s performance by comparing its loss curve
with state-of-the-art QML optimizers.
This comparison includes the Gradient Descent family of optimizers, namely
standard gradient descent with a fixed learning rate, % TODO: cite! and isnt this really stochastic GD?
\texttt{Adam} \cite{kingma_adam_2017} and
\texttt{Adagrad} \cite{duchi_adaptive_2011}.
Additionally, we test our \texttt{Crotosolve} implementation against PennyLane's
implementation of the \texttt{Rotosolve} algorithm
\cite{ostaszewski_structure_2021,bergholm_pennylane_2022}, which is extended by
Wierich's paper on ``General parameter-shift rules for quantum gradients'' to
support further types of gates \cite{wierichs_general_2022}.

To get meaningful results, we chose to look at loss curves of these optimizers
when applied to well-known and widely-used quantum circuits.
Here, we use the set of circuit templates presented in a paper on
``Expressibility and Entangling Capability [...]'' metrics by Sukin Sim et al.
\cite{sim_expressibility_2019}.
As shown in this paper, these circuits have various degrees of expressibility
and entangling capability. % TODO why is that good
% TODO mention that this set of circuits has also been used as a benchmark in
%      other cases

To reduce statistical noise, we generate random initial parameter values
for all evaluated circuits and run each optimizer several times for each pair
of circuit and initial state.
These runs are then averaged and the average loss curves from various optimizers
are combined within a chart for each combination.
% TODO: is this clear enough
Note that we record the loss curve with respect to the number of circuit
evaluations, not the number of iterations or time taken to optimize a circuit's
parameters.
While time is the metric that will be the most interesting long-term,
today's quantum devices and simulators vary heavily in this regard.
For example, gradient-based optimizers benefit from PennyLanes gradient
evaluation optimizations which won't be available on real quantum hardware.
% TODO: cite!
Meanwhile, gradient-free optimizers like \texttt{Crotosolve} and
\texttt{Rotosolve} can't take advantage of these simulation tricks.
Iterations, on the other hand, put both \texttt{Crotosolve} and
\texttt{Rotosolve} in favor since they have to optimize each parameter
individually within each iteration step, which is much more computational work
than a single gradient descent step regardless of the execution environment.
Circuit evaluations, however, are proportional to the execution time on real
quantum hardware as long as the classical work required between evaluations is
negligible.

% TODO describe the device this has been run on

\subsubsection*{Results}

Figure X shows an example loss curve chart, featuring the average results of all
optimizers for a random circuit Y instance. % TODO mention actual numbers
This chart has been selected as it shows all key characteristics described in
the following but many more of these charts can be found in the appendix.
% TODO actually put them in the appendix
% TODO this sounds a lot like results were selected, emphasis should be on this
%      chart showing ALL characterists that can also be found in other charts

Comparison to gradient-based optimizers: 



\subsubsection*{Discussion}

\begin{itemize}
    \item mention complexity, compare number of evaluations
\end{itemize}

\section{Outline}
\begin{itemize}
    \item
        Evaluate the accuracy of the model, the number of steps in the
        optimization loop and the number of circuit evaluations
        \cite{wendenius_gradient-free_2023,ostaszewski_structure_2021}.
    \item
        Compare the results with the performance of other established
        optimizers (e.g., Adam \cite{kingma_adam_2017}, Gradient Descent and
        Quantum Natural Gradient \cite{stokes_quantum_2020}) as well as the
        \emph{Sine Exact} and \emph{Sine Iterative} variants proposed in
        \cite{wendenius_gradient-free_2023}.
        % TODO: sources for other optimizers
        For this purpose, evaluate at least the circuits from
        \cite{sim_expressibility_2019} that were also evaluated in the
        Wendenius et al. paper \cite{wendenius_gradient-free_2023}.
        % TODO: more circuits to evaluate? eileen might have access to
        %       something...
    \item
        Think about the effects of barren plateaus on this optimizer.
        % TODO: do think about this and come up with a concrete task!
        % TODO: cite barren plateaus
\end{itemize}

\chapter{Related work}
\label{chap:related-work}

As discussed in chapter \ref{chap:gradient-free}, our gradient-free optimization
algorithm extends the idea of the \texttt{Rotosolve}
\cite{ostaszewski_structure_2021} algorithm.

While in this thesis, I have presented an extension of the \texttt{Rotosolve}
algorithm for a second type of parameterized gates, a different approach with a
similar outcome has been introduced Wierichs et al..
In ``\emph{\citefield{wierichs_general_2022}{title}}'', they present\dots

TODO:
\begin{itemize}
    \item explain how they use discrete fourier series to reconstruct univariate
        loss functions
    \item Also explain what the spectrum and frequencies are in this context
    \item Name limitations of this approach (preprocessing?)
    \item Explain how pennylane adopts this into their implementation of the
        rotosolve algorithm.
    \item Note how both my approach and this one result in the same method in
        the end (at least if only RP and CRP gates are present)
\end{itemize}. . . 

% ADOPTION
To my knowledge, Pennylane \cite{bergholm_pennylane_2022} is currently the only
major quantum computing SDK implementing a variant of the \texttt{Rotosolve}
\cite{ostaszewski_structure_2021} algorithm.
Given a PQC and the spectrum for each of its parameters, the Pennylane
\texttt{Rotosolve} implementation reconstructs the univariate cost function for
each parameter and optimizes it numerically, similar to our approach from
chapter \ref{chap:proof-of-concept}.
Only for spectra corresponding to pauli rotations, Pennylane's
\texttt{Rotosolve} implementation applies the eponymous analytic methodology
from Ostaszewski's \texttt{Rotosolve} proposal.

\section{Outline}
\begin{itemize}
    \item
        Find more literature on other gradient-free optimization
        approaches.
    \item Generalize algorithm to detect types of gates? (see POC)
\end{itemize}
\chapter{Conclusion}
\label{chap:conclusion}

\section{Outline}
\begin{itemize}
    \item
        Mention how the concept of gradient-free optimization can be
        extended from rotational gates to controlled rotational gates.
    \item
        Mention the proof-of-concept implementation and its evaluation.
    \item
        Conclude whether these contributions are a
        significant improvement to the underlying optimization technique
        and compare its performance to other state-of-the-art optimizers.
    \item
        Present some ideas for future work.
        This might included the extension of this concept to even more
        gates, so that this approach can eventually cover a universal gate
        set to optimize arbitrary PQCs.
        % TODO: unproven guess: If we apply this to \tt{CCX} gates, we might
        %       be able to optimize arbitrary circuits with this approach.
        % note: future work must not be straight-forward and may branch out
        %       into different directions!
        \begin{itemize}
            \item Detect CRP or RP from the code
        \end{itemize}
\end{itemize}


\printbibliography[
    heading=bibintoc
]

% TODO: uncomment if we need an appendix
% \appendix
% \input{sections/??-appendix}

\end{document}

\documentclass[
    twoside,
    english
]{sdqthesis}

% for rendering quantum circuits with tikz
\usepackage{tikz}
\usetikzlibrary{quantikz}

\usepackage{blochsphere}

% for typesetting frequently used fractions with less vertical space (\sfrac)
\usepackage{xfrac}

% for typesetting pseudocode algorithms
\usepackage[
    linesnumbered,
    ruled % algorithm styling similar to booktabs
]{algorithm2e}
%% This declares a command \Comment
%% The argument will be surrounded by /* ... */
\SetKwComment{Comment}{/* }{ */}

\usepackage[citestyle=numeric,style=numeric,backend=biber]{biblatex}
\addbibresource{thesis.bib}

% for arranging multiple figures next to each other
\usepackage{subfig}

\graphicspath{{./images/}}% this command from the graphics package allows to add folders (here subfolder "examplefiles") to be searched for graphic/image files

 % TODO: improve wording and appearance of the title page, maybe mention that this
 %       is an exposé for a bachelor's thesis?
 \author{Maximilian Tim Schweikart}
 \title{Extending gradient-free Optimization of Parameterized Quantum Circuits to Controlled Pauli Gates}
 \thesistype{Bachelor's Thesis}
 \myinstitute{Steinbuch Centre for Computing}
 \grouplogo{../../images/SCC-Logo-notext} % path is relative to the sdqthesis/logos folder!
 \reviewerone{Prof. Dr. Achim Streit}
 \reviewertwo{Prof. Dr. Bernhard Neumair}
 \advisorone{Dr. Eileen Kühn}
 \advisortwo{Dr. Max Fischer}
 \editingtime{Aug 15, 2023}{Dec 15, 2023}
 \settitle
 
\begin{document}
\setpdf
\maketitle
\frontmatter

% TODO: declaration
\setcounter{page}{1}
\pagenumbering{roman}

% TODO: abstract
%\includeabstract

\tableofcontents
% TODO: uncomment once there are any relevant figures and tables
% \listoffigures
% \listoftables

% === content of the thesis ===
\mainmatter

\chapter{Introduction}
\label{chap:intro}
% draft for the introduction to the thesis

%note: eileen recommends ~half a page
%todo: what is this thesis about?
%todo: why is the topic important?
%todo: what do i plan to find out and what is the solution supposed to look like?
%todo: which steps and tasks am I willing to take as part of this thesis?

While the first theoretical foundations of quantum computers have already been
developed in the 1980s, quantum computers have only recently gathered widespread
attention with the development and availability of real quantum computing
hardware \cite{nielsen_quantum_2007,hidary_quantum_2021}.
The field is currently evolving rapidly with new tools, algorithms and hardware
being released every month. % TODO: source?
Still, today's quantum devices are subject to high amounts of noise, have a very
limited number of qubits and are not fully connected, which often further
reduces the number of available qubits and gates needed to perform a given task.
% TODO: source?
% TODO: given? really?

One promising idea is to apply the concept of (classical) machine learning to
quantum computers because (a) machine learning can often be used to approximate
complex functions without much knowledge about their nature and
% TODO: cite! and is this too vague?
(b) machine learning can work with or even benefit from a limited amount of
noise \cite{ciliberto_quantum_2018}.
One common approach to quantum machine learning (QML) is to use a classical
gradient-based optimizer with a parameterized quantum circuit (PQC) to
approximate a given probability distribution.
% TODO: this enters the topic too quickly! PQCs must be explained
However, this approach requires both a lot of shots to calculate the gradient
and a lot of refinement steps to perform the gradient walk.
% TODO: is refinement really the right word for this? (same for gradient walk)

In ``\emph{\citefield{ostaszewski_structure_2021}{title}}''
\cite{ostaszewski_structure_2021}, Ostaszewski et al. explore a different
optimization approach.
Instead of optimizing a circuit through a gradient-based approach, they observed
the sinusoid effect of rotational gate parameters on the output of a PQC and
calculate the optimal parameters analytically by determining the properties of
the sine wave through a small number of circuit evaluations. 
Just like with gradient-based optimizers, this approach does not necessarily
find the global optimum because it optimizes parameters individually.
% TODO: what about the shots?
It has only been tested for optimizing the rotation parameters and directions
for single-qubit rotational gates within arbitrary quantum circuits.
% TODO: that's not true, they just haven't contributed a solution for other
%       types of gates yet! There's even a plot about CRP gates!
However, experiments show good results with this approach compared to
state-of-the-art optimizers like Adam \cite{kingma_adam_2017},
Gradient Descent and Quantum Natural Gradient \cite{stokes_quantum_2020}.
% TODO: cite? and QNG is not a thing any more, the new paper cites SPSA
These promising results naturally pose the question whether this approach can be
extended to other types of gates.

Because of their similarity with the studied rotational gates, this bachelor
thesis attempts to find a similar optimization technique for controlled
rotational gates (i.e., \texttt{CRX}, \texttt{CRY} and \texttt{CRZ}).
After some theoretical backgrounds on quantum computing and QML in general, we
will derive an analytic optimization method for parameters of controlled
rotation gates in PQCs.
We will then present a proof-of-concept implementation%
\footnote{The implementation will be made available with an open-source license
on GitHub}
that extends the \texttt{Rotosolve} optimizer proposed by Ostaszewski et al.
We will evaluate the accuracy and the performance of this optimizer both
analytically and through experiments, comparing it to both the original
implementation for rotational gates and to state-of-the-art QML optimizers.
We will hopefully be able to conclude that our extension can further improve the
gradient-free QML optimizer.

\section{Outline}
\begin{itemize}
    \item
        Explain the motivation and relevance of the topic.
    \item
        Explicate the goals, limits and structure of the bachelor thesis.
    \item
        Name the contributions of this thesis.
    \item
        Provide a realistic outlook on the possible impacts of these
        contributions.
\end{itemize}

\chapter{Background}
\label{chap:background}

\section{A brief introduction to Quantum Computing}
At its core, the operation of a quantum computer revolves around the gates and
qubits.

Qubits (quantum bits) make up the data register of a quantum computer.
Just like a classical bit can be in the $0$ or $1$ state, a qubit can be in the
corresponding $\ket 0$ or $\ket 1$ state.
In addition to these two basis states, however, quantum bits can be in a
superposition of these two states,

$$\ket \psi = \alpha_0 \cdot \ket 0 + \alpha_1 \cdot \ket 1,$$

where the probability amplitudes $\alpha_0, \alpha_1 \in \mathbb C$ can be any
complex numbers with $\left|\alpha_0\right|^2 + \left|\alpha_1\right|^2 = 1$.
This essentially increases the expressibility of a qubit in comparison to a
classical, digital bit from the discrete set $\left\{0, 1\right\}$ to the complex,
two-dimensional sphere surface
$\left\{\vec \alpha \in \mathbb C^2 \mid \left|\vec\alpha\right|^2\right\}$.
This means that a quantum computer with a single qubit can work with continuous,
multidimensional data, while a classical computer with a single bit can only
work with this single bit.

Unfortunalely, however, this multidimensional state of a quantum computer
collapses into a discrete state upon observation.
This means, if we try to find out what state the quantum computer is in, the
superposition will either turn into $\ket 0$ (i.e., $\alpha_0 = 1, \alpha_1 = 0$)
or $\ket 1$ (i.e., $\alpha_0 = 0, \alpha_1 = 1$).
The probability of the qubit collapsing into either of these states is given by
the squared absolute value of its probability amplitude,

$$\mathbb P(M = \ket n) = \left| \alpha_n \right|^2.$$

Thus, the normalization of the probability amplitudes, 
$\sum_{n \in \left\{0, 1\right\}} \alpha_n \ket n$, is in fact a normalization
of the probability distribution over the given basis states.
While we cannot measure $\alpha_0$ and $\alpha_1$ directly, our observation is
still influenced by them.
If we know how to reproduce a state with the same (unknown) amplitudes
$\alpha_1, \alpha_2$, we can measure the same state multiple times.
Doing so many times gives us an empirical probability distribution for the basis
states.
% GLOBAL PHASE
% BLOCH SPHERE

\subsection{Multi-qubit-systems}
The separated state of several qubits can be combined to one multi-qubit state
through the outer product.
With two qubits, this results in

\begin{equation}
    \label{eq:separate-state}
    \begin{split}
        &\left(\alpha_0 \ket 0 + \alpha_1 \ket 1\right) \otimes \left(\beta_0 \ket 0 + \beta_1 \ket 1\right) \\
        = &\alpha_0\beta_0 \ket 0 \otimes \ket 0 + \alpha_0\beta_1 \ket 0 \otimes \ket 1 + \alpha_1\beta_0 \ket 1 \otimes \ket 0 + \alpha_1\beta_1 \ket 1 \otimes \ket 1 \\
        = &\underbrace{\alpha_0\beta_0}_{=: \gamma_{00}} \ket{00} + \underbrace{\alpha_0\beta_1}_{=: \gamma_{01}} \ket{01} + \underbrace{\alpha_1\beta_0}_{=: \gamma_{10}} \ket{10} + \underbrace{\alpha_1\beta_1}_{=: \gamma_{11}} \ket{11}.
    \end{split}
\end{equation}

with $\sum_{n \in \left\{00, 01, 10, 11\right\}} \left|\gamma_n\right|^2 = 1$.

Similarly, every system with $n$ qubits can be described in a state

\begin{equation}
    \label{eq:separate-state-n}
    \ket\psi = \bigotimes_{i=1}^n \ket{\psi_i} = \sum_{j=0^n}^{1^n} \gamma_j \ket j
\end{equation}

with the probability amplitudes
$\gamma_n \in \mathbb C, \sum_n \left|\gamma_n\right|^2 = 1$ for the standard
basis states $\ket{0^n}, \dots, \ket{1^n}$.

\subsection{Gates and circuits}

Unlike with classical bits and gates, it is physically impossible for a quantum
computer to create copies of a qubit.
While classical computers typically read (copy), transform (think add) and write
data from and to registers, quantum computers have to do all operations
in-place.

Quantum states can be changed by applying operators to them.
Since the state of an $n$-qubit-system can be described through its probability
amplitudes $\gamma_i$, we can represent this state with a $2^n$-dimensional
complex vector.
If we identify $\ket 0 \equiv \begin{pmatrix}1 & 0\end{pmatrix}^\top$ and
$\ket 1 \equiv \begin{pmatrix}0 & 1\end{pmatrix}^\top$,
equation \ref{eq:separate-state-n} leaves us with

\begin{equation}
    \ket\psi \equiv \begin{pmatrix} \gamma_{0^n} \\ \vdots \\ \gamma_{1^n}\end{pmatrix}
\end{equation}

implicitly.

Universal quantum computers allow us to apply operations to these states that
are described by unitary matrices\footnote{
    A matrix $A \in \mathbb{C}^{N \times N}$ is called unitary iff its conjugate
    transpose is its inverse, i.e. $\overline{A^\top} = A^{-1}$.
}.
% COMMONLY USED EXAMPLE GATES, PARAMETERIZABLE (CONTROLLED) ROTATIONAL PAULI
%  GATES IN PARTICULAR
% UNIVERSAL GATE SETS
% TODO: MEASUREMENTS

In Quantum Computing, every manipulation of the quantum state can be described
by a so-called gate.
A gate is a unitary transformation of the quantum-state.
Gates can act on one or multiple qubits and we can compose bigger gates by
computing the outer product of gates.

We can visualize the application of gates to a system of qubits through
quantum circuits.
In these circuits, each qubit is displayed as a horizontal line and gates are
displayed as boxes that overlay the lines of the qubits they are applied to.
Measurements are indicated by a box with a meter inside.

Take for example the Hadamard gate $H$,

\begin{equation}
    H = \begin{pmatrix}1 & 1 \\1 & -1\end{pmatrix}\,.    
\end{equation}

Applying the Hadamard gate to a single qubit will transform the
$\ket 0$ state into $\frac{1}{\sqrt 2}\ket 0 + \frac{1}{\sqrt 2}\ket 1$ and
$\ket 1$ into $\frac{1}{\sqrt 2}\ket 0 - \frac{1}{\sqrt 2}\ket 1$, which are
both equal superpositions of the original states.

\begin{figure}[h]
    \label{fig:H-circuit}
    \centering
    \begin{quantikz}
        \lstick{\ket{0}}    & \gate{H}  & \meter\qw
    \end{quantikz}
    \caption{
        In this single-qubit quantum circuit, a Hadamard gate is applied to the
        qubit before it is measured.
        This circuit will probablistically output $\ket 0$ just as often as
        $\ket 1$.
    }
\end{figure}

All single-qubit gates $U \in \mathbb{C}^{2 \times 2}$ can be represented by
products of \emph{pauli rotation gates} $RX(\varphi), RY(\varphi)$ and
$RZ(\varphi)$,

\begin{equation}
    \label{eq:rotational-pauli-gates}
    RP\left(\varphi\right) = \cos\left(\frac\theta2\right) \cdot I_2 - i \cdot \sin\left(\frac\theta2\right) \cdot P,\quad
    P \in \left\{X, Y, Z\right\}
\end{equation}

These gates are called \emph{parameterized gates} since they can be adjusted by a
continuous parameter $\varphi \in \left[0, 2\pi\right]$.
The Hadamard gate, for example, can be written as
$H = RX(\pi) \cdot RY(0.5\pi)$.

Single-qubit operations can be simulated trivially by computing the product of
the gate matrices and the state vector.

The problem gets more interesting with two-qubit gates.
For example, the controlled $X$ gate (also known as $CNOT$) works on a system
of two qubits and maps the basis states as follows:

\begin{equation}
    \label{eq:cx-zbasis}
    \begin{align}
        CX &= \begin{pmatrix}
            1&&&\\
            &1&&\\
            &&&1\\
            &&1&
        \end{pmatrix}
        CX \ket{00} &= \ket{00}\\
        CX \ket{01} &= \ket{01}\\
        CX \ket{10} &= \ket{11}\\
        CX \ket{11} &= \ket{10}
    \end{align}
\end{equation}

When applying the $CX$ gate to states from the computational basis, the second
bit is flipped ($X$ is applied) iff the first bit is $\ket 1$.
States other than those from the computational basis might, however, not follow
this simple principle.
The $\ket{++}$ state, for example, is transformed into the
$\ket{-+} = CX \ket{++}$ state, changing the control qubit but not the target
qubit.

\section{Quantum Machine Learning}

\subsection{Optimization techniques}

\section{Outline}
Background: Quantum computing and QML with PQCs
\begin{itemize}
    \item
        Briefly explain the operation of a quantum computer
        \cite{nielsen_quantum_2007}.
        This section should mention qubits, gates, universal gate sets,
        measurements and their mathematical representation.
        Make sure to mention parameterizable gates like rotational pauli (RP)
        gates and controlled rotational pauli (CRP) gates.
    \item
        Explain the setup for quantum machine learning with parameterized
        quantum circuits (PQCs) \cite{mitarai_quantum_2018}.
        Mention the analogy of quantum machine learning with PQCs with
        classical machine learning setups \cite{bishop_pattern_2006}.
        This section should include a few examples and cite demonstrations
        as well as evaluations of the idea.
    \item
        Go into detail on the different optimization techniques used for
        this approach.
        This section should mention state-of-the-art optimizers such as
        Adam \cite{kingma_adam_2017}, Gradient Descent and
        % TODO: cite gradient descent?
        (Quantum) Natural Gradient \cite{stokes_quantum_2020}.
        % TODO: SPSA instead of QNG?
        Also explain the parameter shift rule
        \cite{mitarai_quantum_2018,schuld_evaluating_2019} as we are trying
        to replace it.
        % TODO: re-formulate
        % TODO: maybe explain fourier sums too
\end{itemize}

\chapter{\texttt{Crotosolve}: Gradient-free controlled rotational Pauli gate optimization}
\label{chap:gradient-free}

In \autoref{chap:background}, I have presented how Quantum Machine Learning
relies on optimizers to gradually minimize the expectation value of a
parameterized quantum circuit.
While many of these optimizers are gradient-based, the gradient-free
\texttt{Rotosolve} optimizer by Ostaszewski et al. treats the different
parameters of PQCs independently \cite{ostaszewski_structure_2021}.
By reconstructing the univariate loss function of a rotational Pauli gate
parameter, the minimizing parameter value can be calculated analytically.

In this chapter, I present \texttt{Crotosolve}, a similar method for parameters
of controlled rotational Pauli gates.
I first show in \autoref{sec:gradient-free:effect} that the univariate loss
curve of a controlled rotational Pauli gate can be described as the sum of two
sinusoidal functions with different frequencies.
To reproduce the concrete loss curve, the amplitudes of these functions as well
as their offsets in $x$- and $y$-direction need to be determined.
In \autoref{sec:constants}, I show how to determine these constants
algorithmically using a minimum number of circuit evaluations.
Finally, I outline how to minimize this reconstruction in
\autoref{sec:minimization}.

\section{Effect}
\label{sec:gradient-free:effect}

\begin{figure}
    \centering
    \begin{quantikz}
        \lstick{\ket{0}}    & \gate[wires=2]{U} & \ctrl{1}          & \gate[wires=2]{V}\slice[style=black]{$\ket{\varphi(\theta)}$}  & \meter\qw \\
        \lstick{\ket{0}}    &                   & \gate{RP(\theta)} & \qw                               & \qw
    \end{quantikz}
    \caption{The analyzed quantum circuit is composed of gates $U$, a controlled
    rotational Pauli gate $CRP(\theta)$, gates $V$ and a single measurement.
    $\ket{\varphi(\theta)}$ denotes the quantum state immediately before the measurement.}
    \label{fig:crp-circuit}
\end{figure}

I analyze the mathematical structure of the loss curve with respect to the
parameter $\theta$ of a single $CRP(\theta)$ gate.
Without loss of generality, I will assume a quantum circuit with only two
qubits and only a single $CRP(\theta)$ gate, as shown in
\autoref{fig:crp-circuit}.
% TODO: generality explanation
Before and after the $CRP(\theta)$ gate, the circuit can contain arbitrary gates
that do not depend on $\theta$.
These gates have been summarized as the $U$ and $V$ gates.
Following the gates, a single qubit is measured.
Note that both the choice of the measured qubit and the choice of the
controlling qubit in the $CRP(\theta)$ gate are irrelevant for this analysis, as
different choices can be covered with this circuit structure by appending or
prepending swap gates to $U$ and $V$.

The possible outcomes of the measurement are distributed randomly and this
distribution depends on $\theta$ since the circuit depends on $\theta$.
The probability of measuring $\ket 0$ in the measured qubit can be computed as
the combination of all outcomes' probabilites where the measured qubit is
$\ket 0$.

\begin{equation}
    \label{eq1}
    \begin{split}
        \mathbb{P}(M_0 = \ket 0 \mid \theta)
            &= \mathbb{P}(M = \ket{00} \mid \theta) + \mathbb{P}(M = \ket{01} \mid \theta)
    \end{split}
\end{equation}

To resolve this result, I will show in \autoref{sec:single-outcome-probability}
that there are $d_1, \dots, d_5 \in \mathbb R$ so that
$\mathbb P(M = \ket\alpha \mid \theta) = d_1 + d_3 \cos\left(\sfrac\theta2 + d_2\right) + d_5 \cos\left(\theta + d_4\right)$.
Using trigonometric identities, equation \ref{eq1} can be simplified to a term
of the same structure \cite{bronstejn_taschenbuch_2016}.
% TODO: "sums of these terms"

\begin{equation}
    \label{eq:total-result-probability}
    \begin{split}
        \mathbb P\left(M_0 = \ket 0 \mid \theta\right)
            &\stackrel{(\ref{eq1})}=
                \mathbb{P}(M = \ket{00} \mid \theta) + \mathbb{P}(M = \ket{01} \mid \theta) \\
            &\stackrel{(\ref{eq:single-prob-simplified-simplified})}=
                d_1^{00} + d_3^{00} \cos\left(\sfrac\theta2 + d_2^{00}\right) + d_5^{00} \cos\left(\theta + d_4^{00}\right) \\
                &\quad + d_1^{01} + d_3^{01} \cos\left(\sfrac\theta2 + d_2^{01}\right) + d_5^{01} \cos\left(\theta + d_4^{01}\right) \\
            &= d_1 + d_3 \cos\left(\sfrac\theta2 + d_2\right) + d_5 \cos\left(\theta + d_4\right)
    \end{split}
    % TODO: janky alignment
\end{equation}

\section{Single outcome probabilites}
\label{sec:single-outcome-probability}

As pointed out in the previous section, I will show in this section that there
are constants $d_1, \dots, d_5 \in \mathbb R$ so that the probability of a
single outcome can be expressed with the following equation.

\begin{equation}
    \mathbb P\left(M = \ket \alpha \mid \theta\right)
        = d_1 + d_3 \cos\left(\sfrac\theta2 + d_2\right) + d_5 \cos\left(\theta + d_4\right)
\end{equation}

To find this equation, I will start with the general equation for the
probability of a single measurement and the definition of the $CRP(\theta)$
gate. 
Successively, I will absorb terms that are independent from $\theta$ into
constants and combine $\sin$ and $\cos$ terms using trigonometric identities
from \cite{bronstejn_taschenbuch_2016}.
In this context, I will call terms constant if they are independent from
$\theta$.
Note that these terms are not constant w.r.t. $\ket\alpha$.

Let $\ket\alpha$ be any state from the basis of the measurement.
The probability of the measurement to result in $\ket\alpha$ can be computed
from its overlap with $\ket{\varphi(\theta)}$, as introduced in
\autoref{sec:quantum-intro}.
Here, $\ket{\varphi(\theta)} = V \cdot CRP(\theta) \cdot U\ket{00}$ denotes the
quantum state just before the measurement.

\begin{equation}
    \label{eq:single-outcome-prob}
    \begin{split}
        \mathbb P\left(M = \ket\alpha \mid \theta\right)
            &= \lvert\braket{\alpha}{\varphi(\theta)}\rvert^2\\
            &= \lvert \bra\alpha V \cdot CRP(\theta) \cdot U \ket0 \rvert^2 \\
            &= \bra\alpha V \cdot CRP(\theta) \cdot U \ket0
                \cdot \overline{\bra\alpha \cdot V \cdot CRP(\theta) \cdot U \ket0} \\
            &= \underbrace{\bra\alpha V}_{=: \bra{\tilde\alpha}} \cdot CRP(\theta)
                \cdot \underbrace{U \ket0 \cdot \bra0 \cdot U^\dagger}_{=: A} \cdot CRP(\theta)^\dagger
                \cdot \underbrace{V^\dagger \ket\alpha}_{=\ket{\tilde\alpha}} \\
            &= \bra{\tilde\alpha} CRP(\theta) \cdot A \cdot CRP(-\theta) \ket{\tilde\alpha}
    \end{split}
\end{equation}

Inserting the matrix definition of the controlled rotational Pauli gate, this
term can be multiplied out further.

\begin{equation}
    \label{eq:single-prob} 
    \begin{split}
        \mathbb P\left(M = \ket\alpha \mid \theta\right)
            &\stackrel{(\ref{eq:single-outcome-prob})}=\bra{\tilde\alpha} CRP(\theta) \cdot A \cdot CRP(-\theta) \ket{\tilde\alpha} \\
            &=\bra{\tilde\alpha}
                \left(\ket 0 \bra 0 \otimes I + \ket 1 \bra 1 \otimes RP\left(\theta\right)\right) \\
                &\quad \cdot A
                \cdot \left(\ket 0 \bra 0 \otimes I + \ket 1 \bra 1 \otimes RP\left(-\theta\right)\right)
                \cdot \ket{\tilde\alpha} \\
            &= \underbrace{\bra{\tilde\alpha} (\ket 0 \bra 0 \otimes I)}_{=: \bra\gamma} \cdot A \cdot \underbrace{(\ket 0 \bra 0 \otimes I) \ket{\tilde\alpha}}_{=: \ket{\delta}} \\
                &\quad + \underbrace{\bra{\tilde\alpha} (\ket 0 \bra 0 \otimes I)}_{= \bra\gamma} \cdot A \cdot (\ket 1 \bra 1 \otimes RP\left(-\theta\right)) \ket{\tilde\alpha} \\
                &\quad + \bra{\tilde\alpha} (\ket 1 \bra 1 \otimes RP\left(\theta\right)) \cdot A \underbrace{(\ket 0 \bra 0 \otimes I) \ket{\tilde\alpha}}_{=: \ket{\delta}} \\
                &\quad + \bra{\tilde\alpha} (\ket 1 \bra 1 \otimes RP\left(\theta\right)) \cdot A \cdot (\ket 1 \bra 1 \otimes RP\left(-\theta\right)) \ket{\tilde\alpha} \\
            &= \bra\gamma A \ket\delta \\
                &\quad + \bra\gamma A \cdot (\ket 1 \bra 1 \otimes RP\left(-\theta\right)) \ket{\tilde\alpha} \\
                &\quad + \bra{\tilde\alpha} (\ket 1 \bra 1 \otimes RP\left(\theta\right)) \cdot A \ket\delta \\
                &\quad + \bra{\tilde\alpha} (\ket 1 \bra 1 \otimes RP\left(\theta\right)) \cdot A \cdot (\ket 1 \bra 1 \otimes RP\left(-\theta\right)) \ket{\tilde\alpha} \\
    \end{split}
\end{equation}

As $\ket1\bra1 \otimes RP(\theta)$ is a block matrix, any product
$\bra x (\ket1\bra1 \otimes RP(\theta)) \ket y$ can be reduced to
$\bra{x^\downarrow} RP(\theta) \ket{y^\downarrow}$ where $\ket{x^\downarrow}$
contains the components from $\ket x$ corresponding to the non-zero matrix
block.
The summands in \autoref{eq:single-prob} can be further simplified by using this
reduction and inserting the matrix definition for $RP(\theta)$ gates, which was
presented in \autoref{eq:rotational-pauli-gates}.
With these transformations, the second summand from \autoref{eq:single-prob}
resolves to a sum of $\sin$ and $\cos$ functions of $\theta$ with frequency
$\sfrac12$.

\begin{equation}
    \label{eq:single-prob-simplification1}
    \begin{split}
            &\quad \bra\gamma A \cdot (\ket 1 \bra 1 \otimes RP\left(-\theta\right)) \ket{\tilde\alpha} \\
            &= \bra{\gamma^\downarrow} A^\downarrow \cdot RP\left(-\theta\right) \ket{\tilde\alpha^\downarrow} \\
            % TODO: this might be false
            &= \bra{\gamma^\downarrow} A^\downarrow \cdot \left(\cos\left(-\sfrac\theta2\right) I - i \sin\left(-\sfrac\theta2\right) P\right) \ket{\tilde\alpha^\downarrow} \\
            &= \cos\left(-\sfrac\theta2\right) \cdot \bra{\gamma^\downarrow} A^\downarrow \cdot I \ket{\tilde\alpha^\downarrow} \\
                &\quad - i \sin\left(-\sfrac\theta2\right) \cdot \bra{\gamma^\downarrow} A^\downarrow \cdot P \ket{\tilde\alpha^\downarrow} \\
            &= \cos\left(\sfrac\theta2\right) \cdot \underbrace{\bra{\gamma^\downarrow} A^\downarrow \cdot I \ket{\tilde\alpha^\downarrow}}_{=: c_1} \\
                &\quad + \sin\left(\sfrac\theta2\right) \cdot \underbrace{i \cdot \bra{\gamma^\downarrow} A^\downarrow \cdot P \ket{\tilde\alpha^\downarrow}}_{=: c_2} \\
            &= \cos\left(\sfrac\theta2\right) \cdot c_1 + \sin\left(\sfrac\theta2\right) \cdot c_2
    \end{split}
\end{equation}

Similarly, the third summand from equation \ref{eq:single-prob} resolves to a
sum of $\sin$ and $\cos$ functions of $\theta$.
While both the second and third summand have a similar mathematical shape, their
amplitude coefficients $c_1, c_2$ and $c_3, c_4$ may be different.

\begin{equation}
    \label{eq:single-prob-simplification2}
    \begin{split}
            &\quad \bra{\tilde\alpha} (\ket 1 \bra 1 \otimes RP\left(\theta\right)) \cdot A \ket\delta \\
            &= \bra{\tilde\alpha^\downarrow} RP\left(\theta\right) \cdot A^\downarrow \ket{\delta^\downarrow} \\
            &= \bra{\tilde\alpha^\downarrow} \left(\cos\left(\sfrac\theta2\right) I - i \sin\left(\sfrac\theta2\right) P\right) \cdot A^\downarrow \ket{\delta^\downarrow} \\
            &= \cos\left(\sfrac\theta2\right) \cdot \bra{\tilde\alpha^\downarrow} I \cdot A^\downarrow \ket{\delta^\downarrow} \\
                &\quad - i \sin\left(\sfrac\theta2\right) \cdot \bra{\tilde\alpha^\downarrow} P \cdot A^\downarrow \ket{\delta^\downarrow} \\
            &= \cos\left(\sfrac\theta2\right) \cdot \underbrace{\bra{\tilde\alpha^\downarrow} I \cdot A^\downarrow \ket{\delta^\downarrow}}_{=: c_3} \\
                &\quad + \sin\left(\sfrac\theta2\right) \cdot \underbrace{\left(-i\right)\cdot \bra{\tilde\alpha^\downarrow} P \cdot A^\downarrow \ket{\delta^\downarrow}}_{=: c_4} \\
            &= \cos\left(\sfrac\theta2\right) \cdot c_3 + \sin\left(\sfrac\theta2\right) \cdot c_4
    \end{split}
\end{equation}

To resolve the fourth term, the same ideas have to be applied multiple times.
The summand is left in a form of $\sin$ and $\cos$ functions, their squares
and a product of $\sin$ and $\cos$. 

\begin{equation}
    \label{eq:single-prob-simplification3}
    \begin{split}
            &\quad \bra{\tilde\alpha} (\ket 1 \bra 1 \otimes RP\left(\theta\right)) \cdot A \cdot (\ket 1 \bra 1 \otimes RP\left(-\theta\right)) \ket{\tilde\alpha} \\
            &= \bra{\tilde\alpha^\downarrow} RP\left(\theta\right) \cdot A^\downarrow \cdot RP\left(-\theta\right) \ket{\tilde\alpha^\downarrow} \\
            &= \bra{\tilde\alpha^\downarrow} \left(\cos\left(\sfrac\theta2\right) I - i \sin\left(\sfrac\theta2\right) P\right) \\
                &\quad\cdot A^\downarrow \cdot \left(\cos\left(-\sfrac\theta2\right) I - i \sin\left(-\sfrac\theta2\right) P\right) \ket{\tilde\alpha^\downarrow} \\
            &= \cos\left(\sfrac\theta2\right)\cos\left(-\sfrac\theta2\right) \bra{\tilde\alpha^\downarrow} I \cdot A \cdot I \ket{\tilde\alpha^\downarrow} \\
                &\quad + \cos\left(\sfrac\theta2\right)\sin\left(-\sfrac\theta2\right) \cdot (-i) \cdot \bra{\tilde\alpha^\downarrow} I \cdot A \cdot P \ket{\tilde\alpha^\downarrow}  \\
                &\quad + \sin\left(\sfrac\theta2\right)\cos\left(-\sfrac\theta2\right) \cdot (-i) \cdot \bra{\tilde\alpha^\downarrow} P \cdot A \cdot I \ket{\tilde\alpha^\downarrow} \\
                &\quad + \sin\left(\sfrac\theta2\right)\sin\left(-\sfrac\theta2\right) \cdot (-i)^2 \cdot \bra{\tilde\alpha^\downarrow} P \cdot A \cdot P \ket{\tilde\alpha^\downarrow} \\
            &= \cos\left(\sfrac\theta2\right)^2 \cdot \underbrace{\bra{\tilde\alpha^\downarrow} A \ket{\tilde\alpha^\downarrow}}_{=: c_5} \\
                &\quad + \cos\left(\sfrac\theta2\right)\sin\left(\sfrac\theta2\right) \cdot \underbrace{i \cdot \bra{\tilde\alpha^\downarrow} A \cdot P \ket{\tilde\alpha^\downarrow}}_{=: c_6} \\
                &\quad + \sin\left(\sfrac\theta2\right)\cos\left(\sfrac\theta2\right) \cdot \underbrace{(-i) \cdot \bra{\tilde\alpha^\downarrow} P \cdot A \ket{\tilde\alpha^\downarrow}}_{=: c_7} \\
                &\quad + \sin\left(\sfrac\theta2\right)^2 \cdot \underbrace{\bra{\tilde\alpha^\downarrow} P \cdot A \cdot P \ket{\tilde\alpha^\downarrow}}_{=: c_8} \\
            &= \cos\left(\sfrac\theta2\right)^2 \cdot c_5 + \sin\left(\sfrac\theta2\right)^2 \cdot c_8 + \cos\left(\sfrac\theta2\right)\sin\left(\sfrac\theta2\right) \cdot \left(c_6 + c_7\right)
    \end{split}
\end{equation}

With equations \ref{eq:single-prob-simplification1},
\ref{eq:single-prob-simplification2}, and \ref{eq:single-prob-simplification3},
equation \ref{eq:single-prob} can be expressed as the following.

\begin{equation}
    \label{eq:single-prob-simplified}
    \begin{split}
        \mathbb{P}\left(M = \ket\alpha \mid \theta\right)
            &\stackrel{(\ref{eq:single-prob})}= \underbrace{\bra\gamma A \ket\delta}_{=: c_9} \\
                &\quad + \bra\gamma A \cdot (\ket 1 \bra 1 \otimes RP\left(-\theta\right)) \ket{\tilde\alpha} \\
                &\quad + \bra{\tilde\alpha} (\ket 1 \bra 1 \otimes RP\left(\theta\right)) \cdot A \ket\delta \\
                &\quad + \bra{\tilde\alpha} (\ket 1 \bra 1 \otimes RP\left(\theta\right)) \cdot A \cdot (\ket 1 \bra 1 \otimes RP\left(-\theta\right)) \ket{\tilde\alpha} \\
            &\stackrel{\substack{(\ref{eq:single-prob-simplification1})\\(\ref{eq:single-prob-simplification2})\\(\ref{eq:single-prob-simplification3})}}=
                c_9 \\
                &\quad + \cos\left(\sfrac\theta2\right) \cdot c_1 + \sin\left(\sfrac\theta2\right) \cdot c_2 \\
                &\quad + \cos\left(\sfrac\theta2\right) \cdot c_3 + \sin\left(\sfrac\theta2\right) \cdot c_4 \\
                &\quad + \cos\left(\sfrac\theta2\right)^2 \cdot c_5 + \sin\left(\sfrac\theta2\right)^2 \cdot c_8 \\
                &\quad + \cos\left(\sfrac\theta2\right)\sin\left(\sfrac\theta2\right) \cdot \left(c_6 + c_7\right) \\
            &= c_9 \\
                &\quad + \cos\left(\sfrac\theta2\right) \cdot \left(c_1 + c_3\right) + \sin\left(\sfrac\theta2\right) \cdot \left(c_2 + c_4\right) \\
                &\quad + \cos\left(\sfrac\theta2\right)^2 \cdot c_5 + \sin\left(\sfrac\theta2\right)^2 \cdot c_8 \\
                &\quad + \cos\left(\sfrac\theta2\right)\sin\left(\sfrac\theta2\right) \cdot \left(c_6 + c_7\right) \\
    \end{split}
\end{equation}

This equation is composed of many $\sin\left(\sfrac\theta2\right)$ and
$\cos\left(\sfrac\theta2\right)$ terms which occur linearly or in quadratic
form.
These products of $\sin\left(\sfrac\theta2\right)$ and
$\cos\left(\sfrac\theta2\right)$ terms can be combined into linear $\sin$ and
$\cos$ with different frequencies.
To calculate these product terms, the following trigonometric identities are
used \cite{bronstejn_taschenbuch_2016}.

\begin{subequations}
    \label{eq:trigonometric-identities}
    \begin{align}
        \cos^2\left(\theta\right)
            &= \frac12 + \frac12 \cos\left(2\theta\right)
            \label{eq:cos-squared} \\
        \sin^2\left(\theta\right)
            &= \frac12 - \frac12 \cos\left(2\theta\right)
            \label{eq:sin-squared} \\
        \cos\left(\theta\right)\sin\left(\psi\right)
            &= \frac12\sin\left(\theta + \psi\right) + \frac12 \sin\left(\theta - \psi\right)
            \label{eq:cos-sin} \\
        a\cos x + b \sin x
            &= sgn(a) \sqrt{a^2 + b^2} \cos\left(x + \arctan\left(-\frac ba\right)\right)
            \label{eq:cos-sum}
    \end{align}
\end{subequations}

Applying these identities to \autoref{eq:single-prob-simplified} leaves the
equation for the probability of a single measurement outcome in the desired
form.

\begin{equation}
    \label{eq:single-prob-simplified-simplified}
    \begin{split}
        \mathbb{P}\left(M = \ket\alpha \mid \theta\right)
            &\stackrel{(\ref{eq:single-prob-simplified})}= c_9 \\
                &\quad + \cos\left(\sfrac\theta2\right) \cdot \left(c_1 + c_3\right) + \sin\left(\sfrac\theta2\right) \cdot \left(c_2 + c_4\right) \\
                &\quad + \cos\left(\sfrac\theta2\right)^2 \cdot c_5 + \sin\left(\sfrac\theta2\right)^2 \cdot c_8 \\
                &\quad + \cos\left(\sfrac\theta2\right)\sin\left(\sfrac\theta2\right) \cdot \left(c_6 + c_7\right) \\
            &= c_9 \\
                &\quad + \cos\left(\sfrac\theta2\right) \cdot \left(c_1 + c_3\right) + \sin\left(\sfrac\theta2\right) \cdot \left(c_2 + c_4\right) \\
                &\quad + \left(\frac12 + \frac12 \cos\left(\theta\right)\right) \cdot c_5 + \left(\frac12 - \frac12 \cos\left(\theta\right)\right) \cdot c_8 \\
                &\quad + \frac12 \sin\underbrace{\left(\frac\theta2 + \frac\theta2\right)}_{=\theta} + \underbrace{\frac12 \sin\left(\frac\theta2 - \frac\theta2\right)}_{= 0} \\
            &= c_9 + \frac{c_5}{2} + \frac{c_8}{2} \\
                &\quad + \cos\left(\sfrac\theta2\right) \cdot \left(c_1 + c_3\right) + \sin\left(\sfrac\theta2\right) \cdot \left(c_2 + c_4\right) \\
                &\quad + \cos\left(\theta\right) \cdot \left(\frac{c_5}{2} - \frac{c_8}{2}\right) + \sin\left(\theta\right) \cdot \frac12 \\
            &= \underbrace{c_9 + \frac{c_5}{2} + \frac{c_8}{2}}_{=: d_1} \\
                &\quad + \cos\left(\sfrac\theta2 + \underbrace{\arctan\left(-\frac{c_2 + c_4}{c_1 + c_3}\right)}_{=: d_2}\right) \cdot \underbrace{sgn\left(c_1 + c_3\right) \sqrt{(c_1 + c_3)^2 + (c_2 + c_4)^2}}_{=: d_3} \\
                &\quad + \cos\left(\theta + \underbrace{\arctan\left(-\frac{\frac12}{\frac{c_5}{2} - \frac{c_6}{2}}\right)}_{=: d_4}\right) \cdot \underbrace{sgn\left(\frac{c_5}{2} - \frac{c_6}{2}\right) \sqrt{\left(\frac{c_5}{2} - \frac{c_6}{2}\right)^2 + \left(\frac12\right)^2}}_{=: d_5} \\
            &= d_1 + \cos\left(\sfrac\theta2 + d_2\right) \cdot d_3 + \cos\left(\theta + d_4\right) \cdot d_5
    \end{split}
\end{equation}

\section{Determining the constants}
\label{sec:constants}
Equation \ref{eq:single-prob-simplified-simplified} describes the effect of the
rotation angle parameter on the expected value of a single qubit.
The simple structure of this formula comes at the cost of five unknown
constants $d_1, \dots, d_5$.
While it is possible to compute those constants from their definitions in
equations \ref{eq:single-prob} - \ref{eq:single-prob-simplified-simplified},
this computation is as computationally expensive as simulating the execution of
the quantum circuit.
% TODO: add a reason for this. something about calculating <alpha|A|beta>

Instead, because of the sinoidal nature of the function, the constants can be
determined through a few evaluations of the quantum circuit.

\begin{equation}
    \label{eq:y}
    \begin{split}
        y(\theta)
            &= d_1 + \underbrace{d_3 \cos(\sfrac\theta2 + d_2)}_{=: y_1(\theta)} + \underbrace{d_5 \cos(\theta + d_4)}_{=: y_2(\theta)}\\
            &= d_1 + y_1(\theta) + y_2(\theta)
    \end{split}
\end{equation}

We can analyze $y_1$ and $y_2$ separately by constructing interferences of $y$
with a phase-shifted version of itself.
Note that
$\cos(\psi + \pi) = -\cos(\psi), \cos(\psi + 2\pi) = \cos(\psi)$.

\begin{equation}
    \label{eq:d1+y2}
    \begin{split}
        y(\theta) + y(\theta + 2\pi)
            &= d_1 + d_3 \cos(\sfrac\theta2 + d_2) + d_5 \cos(\theta + d_4)\\
                &\quad + d_1 + d_3 \cos(\sfrac\theta2 + d_2 + \pi) + d_5 \cos(\theta + d_4 + 2\pi)\\
            &= d_1 + d_3 \cos(\sfrac\theta2 + d_2) + d_5 \cos(\theta + d_4)\\
                &\quad + d_1 - d_3 \cos(\sfrac\theta2 + d_2) + d_5 \cos(\theta + d_4)\\
            &= 2 d_1 + 2 d_5 \cos(\theta + d_4)\\
            &= 2 d_1 + 2 y_2(\theta)\\
        \Rightarrow y_2(\theta) &= \frac12 (y(\theta) + y(\theta + 2\pi) - 2 d_1)
    \end{split}
\end{equation}

Again, the interference of this function with itself can be used to eliminate
$y_2$.

\begin{equation}
    \label{eq:d1}
    \begin{split}
        &\quad y(\theta) + y(\theta + \pi) + y(\theta + 2\pi) + y(\theta + 3\pi)\\
            &= y(\theta) + y(\theta + 2\pi) + y(\theta + \pi) + y((\theta + \pi) + 2\pi)\\
            &\stackrel{\ref{eq:d1+y2}}= 2d_1 + 2d_5\cos(\theta + d_4) + 2d_1 + 2d_5\cos(\theta + \pi + d_4)\\
            &= 2d_1 + 2d_5\cos(\theta + d_4) + 2d_1 - 2d_5\cos(\theta + d_4)\\
            &= 4d_1\\
        \Rightarrow d_1 &= \frac14(y(\theta) + y(\theta + \pi) + y(\theta + 2\pi) + y(\theta + 3\pi))
    \end{split}
\end{equation}

With $d_1$, we can now work with $y_2$ to determine $d_4$ and $_5$.
To do so, we first need to catch an edge case.
If $d_5 = 0$, then $y_2(\theta)$ is $0$ for all angles $\theta$ and $d_4$ can be
chosen arbitrarily.
Since $\sin$ and $\cos$ have distinct zeros, we can check for this condition
with two evaluations $y_2(\theta)$ and $y_2(\theta + \sfrac32\pi)$.

\begin{equation}
    \label{eq:d5-0}
    d_5 = 0 \quad\Leftrightarrow\quad\bigwedge
    \begin{cases}
        0 = d_5 \cos(\theta + d_4) = y_2(\theta) \\
        0 = d_5 \sin(\theta + d_4) = \cos(\theta + d_4 + \sfrac32\pi) = y_2(\theta + \sfrac32\pi)
    \end{cases}
\end{equation}

If $y_2(\theta) = 0$ and $y_2(\theta + \sfrac32\pi) \neq 0$, we can derive
$d_4$ and $d_5$ as follows.
Note that we restrict the zeros of $\cos$ to be $\sfrac12\pi$ or $\sfrac32\pi$
without loss of generality since $\cos$ is $2\pi$-periodic.
We can further eliminate $\sfrac32\pi$ since choosing $\sfrac32\pi$ over
$\sfrac12\pi$ only changes the sign of the function, which can also be chosen
through its amplitude $d_5$.

\begin{equation}
    \label{eq:whateverman}
    \begin{split}
        0 &= y_2(\theta) = d_5 \cos(\theta + d_4) \\
        &\stackrel{d_5 \neq 0}\Rightarrow\quad \theta + d_4 = \sfrac12\pi\\
        &\Rightarrow\quad d_4 =\sfrac12\pi - \theta
    \end{split}
\end{equation}

The corresponding amplitude can then be computed as

\begin{equation}
    \label{eq:whateverman2}
    \begin{split}
        y_2(\theta + \sfrac32\pi)
            &= d_5 \cos(\theta + d_4 + \sfrac32\pi)\\
            &= d_5 \cos(\sfrac12\pi + \sfrac32\pi)\\
            &= d_5 \cos(2\pi)\\
            &= d_5\,.
    \end{split}
\end{equation}

If $y_2(\theta) \neq 0$, we can use the inverse of the $\tan$ function to
compute $d_4$.

\begin{equation}
    \label{eq:d4}
    \begin{split}
        \frac{y_2(\theta + \sfrac32 \pi)}{y_2(\theta)}
            &= \frac{d_5 \cos(\theta + \sfrac32 \pi + d_4)}{d_5 \cos(\theta + d_4)} \\
            &= \frac{\sin(\theta + d_4)}{\cos(\theta + d_4)} \\
            &= \tan(\theta + d_4) \\
        \Rightarrow \theta + d_4
            &= \arctan\left(\frac{y_2(\theta + \sfrac32 \pi)}{y_2(\theta)}\right) \\
        \Rightarrow d_4
            &= \arctan\left(\frac{y_2(\theta + \sfrac32 \pi)}{y_2(\theta)}\right) - \theta
            % TODO reduce to y
    \end{split}
\end{equation}

The computation of $d_5$ then needs no additional circuit evaluations.

\begin{equation}
    \label{eq:d5}
    \begin{split}
        y_2(\theta)
            &= d_5 \cos(\theta + d_4) \\
        \Rightarrow d_5
            &= \frac{y_2(\theta)}{\cos(\theta + d_4)}
            % TODO reduce to y
        % TODO: is this safe? could divide by zero but we have lots of thetas to choose from!
    \end{split}
\end{equation}

A similar approach can be used to determine the remaining constants, $d_2$ and
$d_3$.
Again, we use an interference of $y$ with a phase-shifted version of itself to
find an equation with only the missing constants.

\begin{equation}
    \label{eq:y1}
    \begin{split}
        y(\theta) - y(\theta + 2\pi)
            &= d_1 + d_3 \cos(\sfrac\theta2 + d_2) + d_5 \cos(\theta + d_4)\\
                &\quad - d_1 - d_3 \cos(\sfrac\theta2 + \pi + d_2) - d_5 \cos(\theta + 2\pi + d_4)\\
            &= d_1 + d_3 \cos(\sfrac\theta2 + d_2) + d_5 \cos(\theta + d_4)\\
                &\quad - d_1 + d_3 \cos(\sfrac\theta2 + d_2) - d_5 \cos(\theta + d_4)\\
            &= 2 d_3 \cos(\sfrac\theta2 + d_2)\\
            &= 2 y_1(\theta)\\
        \Rightarrow y_1(\theta) &= \frac12\left(y(\theta) - y(\theta + 2\pi)\right)
    \end{split}
\end{equation}

Similary to \ref{eq:d1+y2} to \ref{eq:d4}, the missing constants can be derived
from this equation.
First, we handle the case where $y_1$ zeroes out and $d_2$ can be chosen
arbitrarily.

\begin{equation}
    \label{eq:d3-0}
    d_3 = 0 \quad\Leftrightarrow\quad \bigwedge
    \begin{cases}
        0 = d_3 \cos(\sfrac\theta2 + d_2) = y_1(\theta) \\
        0 = d_3 \sin(\sfrac\theta2 + d_2) = d_3 \cos(\sfrac\theta2 + d_2 + \sfrac32\pi) = y_1(\theta + 3\pi)
    \end{cases}
\end{equation}

Second, if $y_1(\theta) = 0$ but $y_1(\theta + 3\pi) \neq 0$, $d_4$ and $d_5$ can
be determined by choosing $\sfrac\theta2 + d_2$ as a zero point of $\cos$.

\begin{equation}
    \label{eq:whateverman-the-second}
    \begin{split}
        0 = y_1(\theta) &= d_3 \cos(\sfrac\theta2 + d_2)\\
        \stackrel{d_3 \neq 0}\Rightarrow \sfrac12\pi &= \sfrac\theta2 + d_2\\
        \Rightarrow \sfrac12\pi - \sfrac\theta2 &= d_2
    \end{split}
\end{equation}

\begin{equation}
    \label{eq:whateverman2-the-second}
    \begin{split}
        y_1(\theta + 3\pi)
            &= d_3 \cos(\sfrac\theta2 + d_2 + \sfrac32\pi)\\
            &= d_3 \cos(\sfrac12\pi + \sfrac32\pi)\\
            &= d_3 \cos(2\pi)\\
            &= d_3
    \end{split}
\end{equation}

And third, we can determine $d_2$ and $d_3$ with the use of $\tan$'s inverse if
$y_1(\theta) \neq 0$.

\begin{equation}
    \label{eq:d2}
    \begin{split}
        \frac{y_1(\theta + 3\pi)}{y_1(\theta)}
            &= \frac{d_3 \cos(\sfrac\theta2 + \sfrac32 \pi + d_2)}{d_3 \cos(\sfrac\theta2 + d_2)} \\
            &= \frac{\sin(\sfrac\theta2 + d_2)}{\cos(\sfrac\theta2 + d_2)} \\
            &= \tan(\sfrac\theta2 + d_2) \\
        \Rightarrow \sfrac\theta2 + d_2
            &= \arctan\left(\frac{y_1(\theta + 3\pi)}{y_1(\theta)}\right) \\
        \Rightarrow d_2
            &= \arctan\left(\frac{y_1(\theta + 3\pi)}{y_1(\theta)}\right) - \sfrac\theta2
    \end{split}
\end{equation}

\begin{equation}
    \label{eq:d3}
    \begin{split}
        y_1(\theta)
            &= d_3 \cos(\sfrac\theta2 + d_2) \\
        \Rightarrow d_3
            &= \frac{y_1(\theta)}{\cos(\sfrac\theta2 + d_2)}
    \end{split}
\end{equation}

Therefore, the constants $d_1, \dots, d_5$ can be determined with a total of six
quantum circuit evaluations
$y(\theta), y(\theta + \pi), y(\theta + \sfrac32\pi), y(\theta + 2\pi), y(\theta + 3\pi)$
and $y(\theta + \sfrac72\pi)$.
For convenience, they are summarized in the following.
In cases where the loss function is independent of a parameter $d_i$, the
arbitrary choice has been denoted with an asterisk $*$.

\begin{subequations}
    \label{eq:constants}
    \begin{align}
        d_1 &= \frac14 (y(\theta) + y(\theta + 2\pi) + y(\theta + \pi) + y(\theta + 3\pi))
            \label{eq:constants-d1}\\
        d_2 &=
            \begin{cases}
                *
                    \,, & \textrm{if } y_1(\theta) = 0 = y_1(\theta + 3\pi)\\
                \sfrac12\pi - \sfrac\theta2
                    \,, & \textrm{if } y_1(\theta) = 0 \neq y_1(\theta + 3\pi)\\
                \arctan\left(\frac{y_1(\theta + 3\pi)}{y_1(\theta)}\right) - \sfrac\theta2
                    \,, & \textrm{if }y_1(\theta) \neq 0
            \end{cases}
            \label{eq:constants-d2}\\
        d_3 &=
            \begin{cases}
                0
                    \,, & \textrm{if } y_1(\theta) = 0 = y_1(\theta + 3\pi)\\
                y_1(\theta + 3\pi)
                    \,, & \textrm{if } y_1(\theta) = 0 \neq y_1(\theta + 3\pi)\\
                \frac{y_1(\theta)}{\cos(\sfrac\theta2 + d_2)}
                    \,, & \textrm{if } y_1(\theta) \neq 0
            \end{cases}
            \label{eq:constants-d3}\\
        d_4 &=
            \begin{cases}
                *
                    \,, & \textrm{if } y_2(\theta) = 0 = y_2(\theta + \sfrac32\pi)\\
                \sfrac12\pi - \theta
                    \,, & \textrm{if } y_2(\theta) = 0 \neq y_2(\theta + \sfrac32\pi)\\
                \arctan\left(\frac{y_2(\theta + \sfrac32 \pi)}{y_2(\theta)}\right) - \theta
                    \,, & \textrm{if } y_2(\theta) \neq 0
            \end{cases}
            \label{eq:constants-d4}\\
        d_5 &=
            \begin{cases}
                0
                    \,, & \textrm{if } y_2(\theta) = 0 = y_2(\theta + \sfrac32\pi)\\
                y_2(\theta + \sfrac32\pi)
                    \,, & \textrm{if } y_2(\theta) = 0 \neq y_2(\theta + \sfrac32\pi)\\
                \frac{y_2(\theta)}{\cos(\theta + d_4)}
                    \,, & \textrm{if } y_2(\theta) \neq 0
            \end{cases}
            \label{eq:constants-d5}
    \end{align}
\end{subequations}

with

\begin{subequations}
    \label{eq:measurements}
    \begin{align}
        y_1(\theta) &= \frac12(y(\theta) - y(\theta + 2\pi))\\
        y_1(\theta + 3\pi) &= \frac12(y(\theta + 3\pi) - y(\theta + \pi))\\
        y_2(\theta) &= \frac12(y(\theta) + y(\theta + 2\pi) - 2d_1)\\
        y_2(\theta + \sfrac32\pi) &= \frac12(y(\theta + \sfrac32\pi) + y(\theta + \sfrac72\pi) - 2d_1)
        \,.
    \end{align}
\end{subequations}

\section{Minimizing the reconstruction}
\label{sec:minimization}
With a complete reconstruction, we can now compute the \texttt{CRP} gate
parameter $\theta$ that minimizes the cost function $y(\theta)$.

\textbf{Note:} This section hasn't been verified yet.
Many of these statements are unchecked and some of them might be plain wrong.
% TODO: check and un-wrong these statements!

\subsubsection*{Analytical approach}
Since $y$ is a composition of well-known differentiable functions, we can
attempt to determine its minimum by first finding the zeros of its derivative
$\sfrac{\partial}{\partial \theta}y$.

\begin{equation}
    \begin{split}
        0 &\stackrel!= \frac{\partial}{\partial\theta} y(\theta)\\
            &= -\frac{d_3}{2} \sin(\sfrac\theta2 + d_2) - d_5 \sin(\theta + d_4)\\
            &= -\frac{d_3}{2} (\sin(\sfrac\theta2)\cos(d_2) + \cos(\sfrac\theta2)\sin(d_2))\\
                &\quad - d_5 (\sin(\theta)\cos(d_4) + \cos(\theta)\sin(d_4))\\
            &= -\frac{d_3}{2} (\sin(\sfrac\theta2)\cos(d_2) + \cos(\sfrac\theta2)\sin(d_2))\\
                &\quad - d_5 (2\sin(\sfrac\theta2)\cos(\sfrac\theta2)\cos(d_4) + (2\cos^2(\sfrac\theta2) - 1)\sin(d_4))\\
            &\approx a \cos(\sfrac\theta2 + b) \cdot (c + \cos(\sfrac\theta2))???
    \end{split}
\end{equation}
% TODO: maybe find citation that no analytical solution is known for this problem?

So... what now?

\subsubsection*{Numerical approach}
Alternatively, we can just throw a numerical minimizer at the problem to find
the function's minimum.
Since $y$ is $4\pi$-periodic, it has an infinite amount of minima, so we'll
focus on finding the minimum within $\theta \in [0, 4\pi]$.
Within these boundaries, $y_1$ has a single minimum $x_1$, $y_2$ has two equal
minima $x_{2a}, x_{2b}$ and $y$ has up to two local minima.
The lowest minimum $x_0$ of $y$ is always located between $x_1$ and the minimum
of $y_2$ that is the closest to $x_1$.
Mathematically, this can be expressed as

\begin{equation}
    \begin{gathered}
        x_2 = \underset{x = x_{2a}, x_{2b}}{\operatorname{argmin}} \left|x_1 - x\right|\\
        \min_\theta y(\theta) \in [\min(x_1, x_2), \max(x_1, x_2)]
    \end{gathered}
\end{equation}
% TODO: take care of proximity through the periodic boundaries!

Within this interval, there can only be a single local minimum so any gradient-
based numerical minimizer should be able to find it.

\chapter{Evaluation}
\label{chap:evaluation}

To evaluate the quality of the algorithm proposed in chapter
\ref{chap:gradient-free}, we implement a proof-of-concept optimizer.
We compare our optimizer to a number of other optimizers frequently used in QML
by creating a loss curve benchmark for all optimizers.
The benchmark features parameterizable quantum circuits with various degrees of
expressibility and from various research applications.

\section{Proof of Concept}
To put the approach proposed in chapter \ref{chap:gradient-free} to test,
we implement a proof-of-concept application with PennyLane
\cite{bergholm_pennylane_2022}.
PennyLane is a popular \cite{unitary_fund_team_results_2022} quantum computing
SDK with extensive documentation and a number of QML optimizers already
implemented.
Particularly, PennyLane is currently the only quantum SDK to implement the
\texttt{Rotosolve} optimizer, to which we want to compare our optimizer in
section \ref{sec:optimizer-comparison}.

While the circuits we want to optimize can be composed of arbitrary static
(i.e., non-parameterized) gates, we limit the types of parameterizable gates to
rotational pauli gates and controlled rotational pauli gates.
Additionally, we assume all parameters are independent from each other and used
without any preprocessing (e.g., having a $RP(x^2)$ gate for a parameter $x$).
% TODO comment on whether this is an actual limitation?

For our algorithm to work, we also need to know which parameters belong to which
type of parameterized gate.
As a simple solution, we assume all $RP$ parameters are passed as an array
separate from the $CRP$ parameters array.
It is in theory, however, also possible to detect the type of gate through static analysis
of the circuit, which is beyond the scope of this thesis.

Similar to the \texttt{Rotosolve} algorithm \cite{ostaszewski_structure_2021},
our \texttt{Crotosolve} implementation optimizes each parameter individually.
Each parameter corresponding to an uncontrolled rotational pauli gate can be
optimized following the exact approach presented by Ostaszewski et al..
And for each parameter corresponding to a controlled rotational pauli gate, we
first reconstruct the univariate cost function (see section \ref{sec:constants})
and then use a numerical optimizer to find the minimizing parameter value (see
section \ref{sec:minimization}).
The algorithm pseudocode is stated in listing \ref{alg:crotosolve}.

\begin{algorithm}
    \caption{The \texttt{Crotosolve} algorithm updates parameter individually}
    \label{alg:crotosolve}
    %
    \SetKwFunction{ReconstructCrp}{ReconstructCrp}
    \SetKwFunction{RotosolveUpdate}{RotosolveUpdate}
    \SetKwFunction{MakeUnivariate}{MakeUnivariate}
    %
    \KwData{$prev\_params \in \mathbb R^n$, $circuit$}
    \KwResult{$params \in \mathbb R^n$}
    \BlankLine
    $params \gets copy(prev\_params)$\;
    \For{$p \gets 1$ \KwTo $n$}{
        $uni \gets $ \MakeUnivariate{$circuit$, $params$, $p$}\;
        \eIf{$p$ belongs to controlled gate}{
            $recon \gets $ \ReconstructCrp{$uni$} \Comment*[r]{see section \ref{sec:constants}}
            $next\_params[p] \gets \underset{x \in [0, 4\pi]}{\operatorname{argmin}}\, recon(x)$ \Comment*[r]{using numerical minimizer}
        }{
            $next\_params[p] \gets $ \RotosolveUpdate{$uni$}\;
        }
    }
\end{algorithm}

\section{Optimizer comparison}
\label{sec:optimizer-comparison}
\subsubsection*{Methodology}

We want to examine \texttt{Crotosolve}'s performance by comparing its loss curve
with state-of-the-art QML optimizers.
This comparison includes the Gradient Descent family of optimizers, namely
standard gradient descent with a fixed learning rate, % TODO: cite! and isnt this really stochastic GD?
\texttt{Adam} \cite{kingma_adam_2017} and
\texttt{Adagrad} \cite{duchi_adaptive_2011}.
Additionally, we test our \texttt{Crotosolve} implementation against PennyLane's
implementation of the \texttt{Rotosolve} algorithm
\cite{ostaszewski_structure_2021,bergholm_pennylane_2022}, which is extended by
Wierich's paper on ``General parameter-shift rules for quantum gradients'' to
support further types of gates \cite{wierichs_general_2022}.

To get meaningful results, we chose to look at loss curves of these optimizers
when applied to well-known and widely-used quantum circuits.
Here, we use the set of circuit templates presented in a paper on
``Expressibility and Entangling Capability [...]'' metrics by Sukin Sim et al.
\cite{sim_expressibility_2019}.
As shown in this paper, these circuits have various degrees of expressibility
and entangling capability. % TODO why is that good
% TODO mention that this set of circuits has also been used as a benchmark in
%      other cases

To reduce statistical noise, we generate random initial parameter values
for all evaluated circuits and run each optimizer several times for each pair
of circuit and initial state.
These runs are then averaged and the average loss curves from various optimizers
are combined within a chart for each combination.
% TODO: is this clear enough
Note that we record the loss curve with respect to the number of circuit
evaluations, not the number of iterations or time taken to optimize a circuit's
parameters.
While time is the metric that will be the most interesting long-term,
today's quantum devices and simulators vary heavily in this regard.
For example, gradient-based optimizers benefit from PennyLanes gradient
evaluation optimizations which won't be available on real quantum hardware.
% TODO: cite!
Meanwhile, gradient-free optimizers like \texttt{Crotosolve} and
\texttt{Rotosolve} can't take advantage of these simulation tricks.
Iterations, on the other hand, put both \texttt{Crotosolve} and
\texttt{Rotosolve} in favor since they have to optimize each parameter
individually within each iteration step, which is much more computational work
than a single gradient descent step regardless of the execution environment.
Circuit evaluations, however, are proportional to the execution time on real
quantum hardware as long as the classical work required between evaluations is
negligible.

% TODO describe the device this has been run on

\subsubsection*{Results}

Figure X shows an example loss curve chart, featuring the average results of all
optimizers for a random circuit Y instance. % TODO mention actual numbers
This chart has been selected as it shows all key characteristics described in
the following but many more of these charts can be found in the appendix.
% TODO actually put them in the appendix
% TODO this sounds a lot like results were selected, emphasis should be on this
%      chart showing ALL characterists that can also be found in other charts

Comparison to gradient-based optimizers: 



\subsubsection*{Discussion}

\begin{itemize}
    \item mention complexity, compare number of evaluations
\end{itemize}

\section{Outline}
\begin{itemize}
    \item
        Evaluate the accuracy of the model, the number of steps in the
        optimization loop and the number of circuit evaluations
        \cite{wendenius_gradient-free_2023,ostaszewski_structure_2021}.
    \item
        Compare the results with the performance of other established
        optimizers (e.g., Adam \cite{kingma_adam_2017}, Gradient Descent and
        Quantum Natural Gradient \cite{stokes_quantum_2020}) as well as the
        \emph{Sine Exact} and \emph{Sine Iterative} variants proposed in
        \cite{wendenius_gradient-free_2023}.
        % TODO: sources for other optimizers
        For this purpose, evaluate at least the circuits from
        \cite{sim_expressibility_2019} that were also evaluated in the
        Wendenius et al. paper \cite{wendenius_gradient-free_2023}.
        % TODO: more circuits to evaluate? eileen might have access to
        %       something...
    \item
        Think about the effects of barren plateaus on this optimizer.
        % TODO: do think about this and come up with a concrete task!
        % TODO: cite barren plateaus
\end{itemize}

\chapter{Related work}
\label{chap:related-work}

As discussed in chapter \ref{chap:gradient-free}, our gradient-free optimization
algorithm extends the idea of the \texttt{Rotosolve}
\cite{ostaszewski_structure_2021} algorithm.

While in this thesis, I have presented an extension of the \texttt{Rotosolve}
algorithm for a second type of parameterized gates, a different approach with a
similar outcome has been introduced Wierichs et al..
In ``\emph{\citefield{wierichs_general_2022}{title}}'', they present\dots

TODO:
\begin{itemize}
    \item explain how they use discrete fourier series to reconstruct univariate
        loss functions
    \item Also explain what the spectrum and frequencies are in this context
    \item Name limitations of this approach (preprocessing?)
    \item Explain how pennylane adopts this into their implementation of the
        rotosolve algorithm.
    \item Note how both my approach and this one result in the same method in
        the end (at least if only RP and CRP gates are present)
\end{itemize}. . . 

% ADOPTION
To my knowledge, Pennylane \cite{bergholm_pennylane_2022} is currently the only
major quantum computing SDK implementing a variant of the \texttt{Rotosolve}
\cite{ostaszewski_structure_2021} algorithm.
Given a PQC and the spectrum for each of its parameters, the Pennylane
\texttt{Rotosolve} implementation reconstructs the univariate cost function for
each parameter and optimizes it numerically, similar to our approach from
chapter \ref{chap:proof-of-concept}.
Only for spectra corresponding to pauli rotations, Pennylane's
\texttt{Rotosolve} implementation applies the eponymous analytic methodology
from Ostaszewski's \texttt{Rotosolve} proposal.

\section{Outline}
\begin{itemize}
    \item
        Find more literature on other gradient-free optimization
        approaches.
    \item Generalize algorithm to detect types of gates? (see POC)
\end{itemize}
\chapter{Conclusion}
\label{chap:conclusion}

\section{Outline}
\begin{itemize}
    \item
        Mention how the concept of gradient-free optimization can be
        extended from rotational gates to controlled rotational gates.
    \item
        Mention the proof-of-concept implementation and its evaluation.
    \item
        Conclude whether these contributions are a
        significant improvement to the underlying optimization technique
        and compare its performance to other state-of-the-art optimizers.
    \item
        Present some ideas for future work.
        This might included the extension of this concept to even more
        gates, so that this approach can eventually cover a universal gate
        set to optimize arbitrary PQCs.
        % TODO: unproven guess: If we apply this to \tt{CCX} gates, we might
        %       be able to optimize arbitrary circuits with this approach.
        % note: future work must not be straight-forward and may branch out
        %       into different directions!
        \begin{itemize}
            \item Detect CRP or RP from the code
        \end{itemize}
\end{itemize}


\printbibliography[
    heading=bibintoc
]

\appendix
\chapter{Appendix}
\label{chap:appendix}

In section \ref{sec:optimizer-comparison}, I presented a characterization of
the \texttt{Crotosolve} loss curve in comparison to other optimizers used in
quantum machine learning.
While the analysis was illustrated with a single example, this section contains
the loss curves of further PQCs.
The circuit ids referenced in the captions refer to the circuit template numbers
from \cite{sim_expressibility_2019}.
All tests use circuits with four qubits and three template layers.
The source code for the dataset generation and visualization is available with
the \texttt{Crotosolve} implementation \cite{crotosolve}.

\newcommand{\losscurve}[1]{%
    \subfloat[Circuit #1 with 4 qubits and 3 layers]{
        \includegraphics[width=0.5\textwidth]{loss-curve_sim#1_4x3.pdf}
    }
}

\begin{figure}
    \centering
    \losscurve{01}
    \losscurve{02}
    \caption{Caption goes here}
\end{figure}
\begin{figure}\ContinuedFloat
    \centering
    \losscurve{03}
    \losscurve{04}
    \hspace{0mm}
    \losscurve{05}
    \losscurve{06}
    \hspace{0mm}
    \losscurve{07}
    \losscurve{08}
    \caption{Caption goes here}
\end{figure}
\begin{figure}\ContinuedFloat
    \centering
    \losscurve{09}
    \losscurve{10}
    \hspace{0mm}
    \losscurve{11}
    \losscurve{12}
    \hspace{0mm}
    \losscurve{13}
    \losscurve{14}
    \caption{Caption goes here}
\end{figure}
\begin{figure}\ContinuedFloat
    \centering
    \losscurve{15}
    \losscurve{16}
    \hspace{0mm}
    \losscurve{17}
    \losscurve{18}
    \hspace{0mm}
    \losscurve{19}
    \caption{Caption goes here}
\end{figure}

\end{document}
